% Created 2013-07-31 水 11:35
\documentclass[11pt]{article}
\usepackage[utf8]{inputenc}
\usepackage[T1]{fontenc}
\usepackage{fixltx2e}
\usepackage{graphicx}
\usepackage{longtable}
\usepackage{float}
\usepackage{wrapfig}
\usepackage{soul}
\usepackage{textcomp}
\usepackage{marvosym}
\usepackage{wasysym}
\usepackage{latexsym}
\usepackage{amssymb}
\usepackage{hyperref}
\tolerance=1000
\providecommand{\alert}[1]{\textbf{#1}}

\title{glz}
\author{MASUDA Akira}
\date{31 7月 2013}

\begin{document}

\maketitle

\setcounter{tocdepth}{3}
\tableofcontents
\vspace*{1cm}
\section{2013-05-28}
\label{sec-1}
\subsection{マヌケな瞬間100連発!}
\label{sec-1_1}

***
【マヌケな瞬間100連発!】
どすけべなお店で、他の客と一緒に待っているとき、

「お客さん、オプションどうされますかぁ?」
と訊かれる。

「え、えと、メガネとパンストの白で、」

ハラの座っている他の客は、(そうかそうか) という目でこちらを見てくる。
\section{2012-12-12}
\label{sec-2}
\subsection{人間なんてラララ}
\label{sec-2_1}

【人間なんてラララ】
***
ヨメサンといいところでチビに乱入される
***
\section{2012-10-30}
\label{sec-3}
\subsection{ちょっとかっこいい運動}
\label{sec-3_1}

***
\section{2012-10-18}
\label{sec-4}
\subsection{ちょっと好きな人}
\label{sec-4_1}
\subsection{}

***
96歳男性に第二子誕生
ハルクホーガン セックス映像流出
ウィルス感染により殺害予告

遠隔操作ウィルス感染によりGLZ乗っ取り企てを掲示板に投稿される
誤認逮捕
ips細胞で髪がフサフサに
ips細胞でナニ毛がフサフサに
まいんちゃんに彼氏ができたという話題
\section{2012-09-27}
\label{sec-5}
\subsection{【女子力高いの、めちゃうぜぇ】}
\label{sec-5_1}
\subsubsection{}

飲み会のときにもう使っていない電源が切れたケータイを持って行って。

自分のドジっ子ぶりを「かわいいアピール」に変える女。

普段はスマホでLINEしてる。
\subsubsection{}

「かわいいiPhoneケース」もらっちゃったから、iPhoneにした、と云う オンナ!!
\subsubsection{}

男性が一人でも参加している飲み会では、酒も煙草も我慢する。
\subsubsection{}

コンビニで
「レジのネーチャン、マジ巨乳だから見てきたほうがいいよ」
と教えてくれるウチのヨメ、マジ男子力 たけぇ

***

飲み会で、
「りとるぐれいさんて、いっつも愛妻弁当ですよね〜」てにこやかに云われたので、
「ヨメに弁当を作ってもらったことなど一度も無いし、チビの幼稚園の弁当も毎朝作ってるし、
  毎朝ご飯作っているのは俺だ」と云うと微妙な顔をしていた。

「40過ぎの男が、女子力高いの、めちゃうぜぇ」て、
たぶん思われたはず。。。 傷心の僕にお食事券ください。
\section{2012-08-15}
\label{sec-6}
\subsection{東京ローカル情報}
\label{sec-6_1}
\section{2012-07-03}
\label{sec-7}
\subsection{オッサントランスレーション}
\label{sec-7_1}

***
サビオ
ママレモン
\section{2012-06-25}
\label{sec-8}
\subsection{ワイルド自慢}
\label{sec-8_1}

***
叔父が酔っぱらったときよくする、

「昔はからずも子どもができてしまった時の対応方法」

がワイルドすぎて衝撃的だったのですが、
とてもここ書けるようなものではありません。
\section{2012-06-05}
\label{sec-9}
\subsection{【セコテク】}
\label{sec-9_1}
\subsubsection{【セコテク】}
\label{sec-9_1_1}

会社の飲み会にて、その会自体は楽しいのだけれど。

お開き後、普段話をしないような人と電車の方向が同じばっかりに、一緒の電車で
当たり障りの無い話を続けるのが苦痛この上なく。

飲み会会場で解散した後は「酔い覚まし」と称して隣駅まで歩きます。
だいたい毎回。
***
3歳のチビは僕とヨメサンから、常にオムツの耐久テストをされている状態です。
馬鹿になりません日々のオムツ代。

いずれ僕が彼から介護を受けるようになったときには、
きっと同様に、耐久テストをされるのだと思います。

でもそれは仕方がないのです。
\subsubsection{【軽く傷つくストーリー】 ヽ(゚∀゚)ノ}
\label{sec-10_1}


5歳のチビが書いた絵を見てヨメサンが云った。
「うまいねー、波平さん?」

チビ「おとうさんだよ」

俺「・・・髪がもうすこしあってもいいんじゃないかな?」

そこそこキズつきました。
\section{2012-04-10}
\label{sec-10}

**軽く傷つくストーリー
\section{2012-04-02}
\label{sec-11}
\subsection{男と女のアンケート}
\label{sec-11_1}

【男と女のアンケート】

・ラブホの廊下で父さん母さんに会ったことがある

・キスマークをファンデーションで隠して出勤したが、気がつけば落ちていた

・ラブホで余ったコンドーさんを持って帰ったことがある
・同じくラブホでティッシュボックスを持って帰ったことがある
・同じくラブホでハミガキセットを持って帰ったことがある

・ラブホで、シャワーヘッドが実家のものと同じだったことに衝撃を受けたことがある

・彼女の残っている蒙古斑に萎えてしまったことがある
\section{2012-01-16}
\label{sec-12}
\subsection{私もできる // GL偉人伝}
\label{sec-12_1}
\subsubsection{memo}
\label{sec-12_1_1}

誰にでもできるんじゃないの、ということ
\subsubsection{\textbf{TODO}}
\label{sec-12_1_2}

【私もできる】

中学校の頃のこと。
先輩二人がどういう流れからか、「俺だってバック転できるぜ!」という言い合いになり。
その現場に居合わせました。

よせばいいのに校内のコンクリの上でロンダードからのバック転。

は見事に失敗。

その先輩はしばら後頭部に大きなガーゼ、と頭にネット。
廊下で見かける度にひそかに失笑してました。

今思えば後頭部強打とかけっこうヤバイすよね…
\section{2012-01-11}
\label{sec-13}
\subsection{GLZ 銭の魂 // 業界クイズ}
\label{sec-13_1}
\subsubsection{memo}
\label{sec-13_1_1}

【GLZ 銭の魂】
ケチ話し、散財、
東電、株、FX、
日銀、レシート、財布
お金大好きエピソード、だって
\subsubsection{\textbf{DONE}}
\label{sec-13_1_2}

【GLZ 銭の魂】

お金の話でいいのかな。

就職前、地元専門学校生の頃、年末年始に集中してバイトしてお金を溜めました。

電気屋で「よしこれだ!!!」と気に入って目をつけていたコンポを、ついに今日購入しようと学校から帰ってきたところ。

何やら家の中の様子がおかしい、裏手キッチン側のガラス窓が割れている。
部屋の中には土足の足跡が点々と…

泥棒に入られたのでした。

現金だけ物色していったようです。
幸いにして、家にお金をあまり置いておかなかった親たち、姉には被害がほとんど無く。

しかし、僕の部屋の机引き出しにしまってあったバイト台は、キッチリ持って行かれました。
おおよそ20万円ナリ…

あれから僕は笑わない子になりました。
\subsubsection{\textbf{TODO}}
\label{sec-13_1_3}
\subsection{サラドレ}
\label{sec-13_2}
\subsubsection{もち豚キムチ 読まれた ヽ(゚∀゚)ノ}
\label{sec-13_2_1}

この曲がかかるともう、反射的に泣けます。子供やヨメサンの顔がふいに思い浮かんで
しまいます。今日も泣かせてください。

福山の「家族になろうよ」だったんだが、1位ではなく…
\section{2012-01-04}
\label{sec-14}
\subsection{新年早々やる気なし… // お楽しみに!}
\label{sec-14_1}
\subsubsection{\textbf{DONE}}
\label{sec-14_1_1}

【新年早々やる気なし…】

今日から仕事です。
休みも大して長くなかったし。
ヨメサンの実家に帰ったのですがあまり正月気分も盛り上がりませんでした。
本当に今は人類滅亡説が囁かれる2012年になったのでしょうか?

義姉が食べすぎで入院したので、彼女にはきっと正月が来たのでしょう。
\subsubsection{\textbf{TODO}}
\label{sec-14_1_2}
\section{2011-12-28}
\label{sec-15}
\subsection{GLZ大感謝祭~メール沢山読みます! // 今年のテレビ人気投票}
\label{sec-15_1}
\subsubsection{}

【GLZ大感謝祭-メール沢山読みます!】
\section{2011-12-27}
\label{sec-16}
\subsection{心の大掃除 // リスナーのあなたに電話をつなぎます!}
\label{sec-16_1}
\subsubsection{memo}
\label{sec-16_1_1}

【心の大掃除】

こころの傷を消したいとか、
片付けたいのはおまえ

心のもやもや
\subsubsection{\textbf{TODO}}
\label{sec-16_1_2}

【心の大掃除】
\section{2011-12-21}
\label{sec-17}
\subsection{今年言われた一言 //有名人~私のビッグニュース2011~}
\label{sec-17_1}
\subsubsection{\textbf{DONE}}
\label{sec-17_1_1}

【今年言われた一言】

「おとーさん、青春ってなに?」
\subsubsection{\textbf{DONE}}
\label{sec-17_1_2}

【今年言われた一言】

つい昨日のこと。
会社の健康診断に行ったのですが。

視力検査のとき、
「その眼鏡は近視用ですか?」
と訊かれました。
四十になると老眼かも、と思われるのですか?
\subsubsection{\textbf{TODO}}
\label{sec-17_1_3}
\section{2011-12-20}
\label{sec-18}
\subsection{うっとうしい酔っ払い一斉摘発! // お楽しみに!}
\label{sec-18_1}
\subsubsection{\textbf{DONE}}
\label{sec-18_1_1}

【うっとうしい酔っ払い一斉摘発!】

とにかく触ってくる女。
勘違いするってば。
\subsubsection{\textbf{DONE}}
\label{sec-18_1_2}

地元に大学があるので、大学生たちがたむろして奇声を発っしているのを見ると。

「てめえら親の金で飲んでクダ巻いてんのかよ!!」

とも云えず。遠目に見てます
\subsubsection{\textbf{TODO}}
\label{sec-18_1_3}

キャバ
ジェントルマン
いま考えればうっとうしい客になっておけば良かった。
チチくらい揉めたかもしれない。
\section{2011-12-19}
\label{sec-19}
\subsection{来年はもうやめよう討論会 // お楽しみに! 2011年末川柳}
\label{sec-19_1}
\subsubsection{memo}
\label{sec-19_1_1}

【来年はもうやめよう討論会】

さずかり婚いうな

森三中のキャッツアイ。
毎回暴れたくなるので。

アルファベット3文字のアイドルグループ

女の子が着てる、マタギみたいなベスト

ぽぽぽぽーん

あしだまな(かけない)

ラブ注入

熊田曜子の女磨き
\subsubsection{\textbf{TODO}}
\label{sec-19_1_2}

よもや、帰宅難民状態に2度も遭遇するとは思いませんでした。
台風の時、身動き取れず入ったラーメン屋で GLZ がかかっていました。
\subsubsection{\textbf{DONE}}
\label{sec-19_1_3}

おべんとう男子、草食系
\subsubsection{\textbf{DONE} ヽ(゚∀゚)ノ 32回目}
\label{sec-19_1_4}

さずかり婚いうな
\subsubsection{\textbf{TODO}}
\label{sec-19_1_5}

2011年末川柳

賞味期限 切れた芸人 ばかり呼び

帰省ラッシュ はまって分かる

年越しそば

曙が マットに沈む スローモーション

年賀状 しまった場所から へそくりが

忘年会

大晦日 ゲ○にまみれて 元旦に

紅白に
\section{2011-12-15}
\label{sec-20}
\subsection{自虐自分クイズ // 年末やっちまったランキング}
\label{sec-20_1}
\subsubsection{\textbf{TODO}}
\label{sec-20_1_1}

【自虐自分クイズ】

20歳台後半のことだったと思います。
まだ自分の給料を自分を自分の自由に使えていたころのこと。

飲み会終り、一人で駅向っていた途中。
しつこい客引きに合って「一万円ぽっきりだから」という甘い言葉に誘われて
ノコノコついていった挙句どうなったでしょう。

1番:
連れて行かれたのは、業務用っぽい建物の地下に続く階段のさらに下。

2番:
その狭い暗闇の中にソファが置いてあり、女性が一人。
今思うとマツコデラックスを少し痩せさせた感じ。

3番:
しかたなくソファに座るとなんやらサワサワしてくるマツコ
服はすべて着たまま。

4番:
そして「この先に進むには、さらに1万いただきます」と云いやがるマツコ

5番:
「やられたな」と思いつつこのままでは帰れず、差し出す一万円

6番:
おざなりに処理され、あっさり終了。
「またねー」とマツコ

7番:
入口で「まいどー」という客引きに

「テメェ!! ぽっきりって云ったよなぁ!! しかもなんだよあのサービスは!!! そしてマツコ!!!!!」

とも云えず、すごすごと家路へ。

【答え】
全部。
\subsubsection{\textbf{TODO}}
\label{sec-20_1_2}

年末やっちまったランキング
\section{2011-12-14}
\label{sec-21}
\subsection{もしもツイッター // 歌詞イメージアンケート2011ファイナル}
\label{sec-21_1}
\subsubsection{memo}
\label{sec-21_1_1}

☆今日のテーマは 「もしもツイッター」 ☆もしもあの人がツイッターをやっていたら… もしもの「つぶやき」を送って下さい☆
海老蔵『テキーラなう』
\subsubsection{\textbf{DONE}}
\label{sec-21_1_2}

「もしもツイッター」

小島慶子「貧乳なう」
「ハロワなう」

せんたみつお「銀座なう!」
ライムスター宇多丸「増毛なう!」
タモさん「暗闇でこけたなう!」

別所哲也「お歳暮はハム!」
中山明日実「炎上なう」
\subsubsection{\textbf{TODO}}
\label{sec-21_1_3}
\section{2011-12-13}
\label{sec-22}
\subsection{見た目悪い人会議 // 日常の不思議ランキング}
\label{sec-22_1}
\subsubsection{memo}
\label{sec-22_1_1}

【見た目悪い人会議】
☆見た目が悪くて損した事、こんな目に遭いました
オールバック
ハラ、めつき、口元
化粧濃い
\subsubsection{\textbf{DONE}}
\label{sec-22_1_2}

【見た目悪い人会議】

耳のピアスと鼻のピアスがチェーンで繋がっていて、そのチェーンに鍵をぶら下げている男の子(20台前半というところ) がいた。

「それって家の鍵?」と訊くと。

「ちげーよ、バイクのだよぉ!」

と怒られた… 知るかっ
\subsubsection{\textbf{DONE}}
\label{sec-22_1_3}

オフィスに来るグリコのおばちゃん。
いつも暗い瞳です。

どんな生活をしているんだろうか… とか考えてしまいます。

結婚生活がうまく行っていないのかな、とか
旦那は薄給で、ギャンブルばっかりしているんだろうな、とか
夜の営みなんてもう20年もしていないんだろうな、とか
今日の朝食のオカズもふりかけだけだったわ、とか

それでイヤイヤ、グリコのおばちゃんやっているんだろうなぁ。
\subsubsection{\textbf{DONE}}
\label{sec-22_1_4}

今のマンション購入のとき、心なし周りの他の人たちより、熱心に薦められた気がします。
お金持ってなさそうに見えたのかも、、ぼくら夫婦は。
\subsubsection{\textbf{DONE}}
\label{sec-22_1_5}

会社に、ロッカーのバイトくんがいるのですが、なぜかヤギヒゲです。
やめろって云いたくて仕方がないんですが。
まあ、ロッカーだし。みたいな。
\subsubsection{\textbf{TODO}}
\label{sec-22_1_6}
\subsubsection{\textbf{TODO}}
\label{sec-22_1_7}

【日常の不思議ランキング】

ひとりでに動く汁椀。

自然発火するブログ

いつの間にか結ばっているイヤフォンコード

いい年なのに膝上スカートを履くおばはん

いい年なのに紫アイシャドーのおばはん

いい年なのにブーツのおばはん

いい年なのにツインテールのおばはん
\subsubsection{\textbf{DONE} ヽ(゚∀゚)ノ でも聞けず}
\label{sec-22_1_8}

教室の天井に刺さっている画鋲
\section{2011-12-12}
\label{sec-23}
\subsection{ほっとしよう日本 // お楽しみに!}
\label{sec-23_1}
\subsubsection{memo}
\label{sec-23_1_1}

宴会で部長の手品が
いたそうで、痛くないこと床屋で
\subsubsection{\textbf{DONE}}
\label{sec-23_1_2}

【ほっとしよう日本】

帰ったらヨメも子ども寝てたとき。
\subsubsection{\textbf{TODO}}
\label{sec-23_1_3}
\section{2011-12-08}
\label{sec-24}
\subsection{人の家でびっくり! // 歌詞につっこもう!}
\label{sec-24_1}
\subsubsection{\textbf{TODO}}
\label{sec-24_1_1}
\section{2011-12-07}
\label{sec-25}
\subsection{東京ミニマム情報 // ツイッタークイズ}
\label{sec-25_1}
\subsubsection{\textbf{DONE}}
\label{sec-25_1_1}

【東京ミニマム情報】

大塚駅近くの『ホープ軒』は店長が若いです。
たまに店長と古参スタッフとの あつれき が見えます。
カウンターに座ってイヤフォンは耳に入れてますが、スイッチは入れず観察してます。
\subsubsection{\textbf{DONE}}
\label{sec-25_1_2}

数日前ですが新宿駅ホームで電車の乗り換えをしていたら、

「\ldots{}たいへん危険ですので、これからの乗車はご遠慮くだ..  オイ!!!」

て聞こえてきました、アナウンス。
\subsubsection{\textbf{TODO}}
\label{sec-25_1_3}

【ツイッタークイズ】

かばちゃんなんと?

「でぃー おー えす」だ! ドスって云うな!
\section{2011-12-06}
\label{sec-26}
\subsection{一言劇場ムリムリ編 // 2011年自分に関係のないニュース人気投票}
\label{sec-26_1}
\subsubsection{memo}
\label{sec-26_1_1}

16時30分から【GROOVE LINE Z】
ピストン西沢 \& 中山明日実☆テーマは「一言劇場〜ムリムリ編〜」 ☆いや、それはちょっと無理です… 言ってしまう光景を、一言劇場にして送って下さい☆ゲストは Every Little Thing
\subsubsection{\textbf{DONE}}
\label{sec-26_1_2}

【一言劇場 ムリムリ編】
「システムエンジニアやってる」と云うと、
パソコン見て、はまだいいんだけど。TV直してとか、ケータイの調子が良くないんだけど、
とか云われます。電気屋じゃないって。
\subsubsection{\textbf{TODO}}
\label{sec-26_1_3}
\section{2011-12-05}
\label{sec-27}
\subsection{うんざりフェス // 今考えるとすごいことランキング}
\label{sec-27_1}
\subsubsection{memo}
\label{sec-27_1_1}
\subsubsection{\textbf{DONE}}
\label{sec-27_1_2}

聞き流すだけで英語が身につく。ってヤツ。
\subsubsection{\textbf{DONE}}
\label{sec-27_1_3}

「オレは睡眠時間が短い」というよく分からない自慢
「昨日徹夜しちゃったんだよねー」とか
\subsubsection{\textbf{DONE}}
\label{sec-27_1_4}

昔やってたバイトの自慢をしてくる。
飲み屋で変なグラスの持ち方して、「俺バイトしてたらグラスいっぱい持てるんだ」とか。
普通に持てよ、てな感じ。

よく聞いてみるとそのバイト3ヶ月で止めてるし、他のバイト一切やってないとか。
\subsubsection{\textbf{DONE}}
\label{sec-27_1_5}

「今月忘年会が二桁入っててさー」
はいはい人気者だね
\subsubsection{\textbf{TODO}}
\label{sec-27_1_6}
\section{2011-12-01}
\label{sec-28}
\subsection{2011年何でもTOP3 // ネットで調べた海外赤っ恥体験}
\label{sec-28_1}
\subsubsection{\textbf{TODO}}
\label{sec-28_1_1}

【2011年何でもTOP3】
\section{2011-11-30}
\label{sec-29}
\subsection{GLZ すんごい住宅情報 // 川柳道場破り}
\label{sec-29_1}
\subsubsection{\textbf{TODO}}
\label{sec-29_1_1}

【すんごい住宅情報】
\section{2011-11-29}
\label{sec-30}
\subsection{恋の勘違い~チェリー編 //超高級品カタログ}
\label{sec-30_1}
\subsubsection{memo}
\label{sec-30_1_1}

ボディタッチが多い。

【1】仲間同士で遊んでいるとき、スキンシップを取ろうとする。
【2】よく目が合う。
【3】旅行土産など些細な贈りものをくれるが、他の人と品物や包みが違う。
【4】ふたりでいるときは、名前で呼んでくれる。
【5】ふたりきりになると、口調がかわいらしくなる。
【6】恋愛に関する悩み相談をしてくる。
【7】頻繁にメールをくれる。
【8】仲間同士で遊んでいるとき、自分にだけちょっと多く話しかけてくる。
【9】急な誘いにも付き合ってくれる。

自分にだけ優しい
\subsubsection{\textbf{DONE}}
\label{sec-30_1_2}

【恋の勘違い~チェリー編】

自分の飲んだジュースを「ちょっと飲ませて」と云いながら口をつける。
\subsubsection{\textbf{DONE}}
\label{sec-30_1_3}

自分の名字と好きな子の名前を組み合わせて紙に名前を書いてみる。
脳内結婚。
\subsubsection{\textbf{DONE}}
\label{sec-30_1_4}

二人っきりになると急に名前で呼ぶ。
\section{2011-11-28}
\label{sec-31}
\subsection{せっかちくん // ダジャレ大国日本}
\label{sec-31_1}
\subsubsection{\textbf{DONE} ヽ(゚∀゚)ノ 31回目}
\label{sec-31_1_1}

ファミレスのテーブルのボタンをがしがし押すと。
時間差で何人か店員さんが来たりします。
すみません、てなります。
\subsubsection{\textbf{DONE}}
\label{sec-31_1_2}

【せっかちくん】

男性トイレの使用に関しまして。
オッサンになるとセットポジションに立つ手前でチャックを下ろします。

これは年齢が上の人ほど顕著になります。
70歳くらいのジイサンになると、トイレのエリアに入った時点でチャックに手をかけ、
セットポジションに立つ時点では完全に露出しております。

ある意味、大変効率的ですね。
自分も今年で40。あんなジイサンになりたいです。
\section{2011-11-24}
\label{sec-32}
\subsection{自己覚醒スペシャル // 大好きヤフー知恵袋}
\label{sec-32_1}
\subsubsection{memo}
\label{sec-32_1_1}

【自己覚醒スペシャル】

前半:自己覚醒スペシャル あなたが目を覚まし、 これからはこうします!!という目標など送ってください。 沢山のメールお待ちしています!

これからは、
朝おきない
あだなで呼ぶ

覚醒するぞっ!
目標、あほな目標とは?

もういい年なので、さいきん今更覚醒するといったことはありません。
小学生くらいの頃を思い出せば、モモレンジャーが敵に捕まってしまい荒縄で縛られて「う、うーん」とか云っていた場面では、きっとなにか覚醒したと思います。
\subsubsection{\textbf{DONE}}
\label{sec-32_1_2}

【自己覚醒スペシャル】

網タイツを見てもエロいことを考えないようします (血涙
\subsubsection{\textbf{DONE}}
\label{sec-32_1_3}

「覚醒する」「なにか今後の目標を」と考えてたのですが、
なかなか思いつかず、自分の悪いところを一覧に挙げてみました。
たくさんあって、軽く落ち込みました。

一番上にあったのは「仕事中、ついラジオに投稿してしまう」でした。
\subsubsection{\textbf{TODO}}
\label{sec-32_1_4}

子どもの頃、モモレンジャーが敵に捕まってしまい、荒縄で縛られて「う、うーん」とか云っていた場面では、きっとなにか覚醒してると思います。
何かは分かりません。
\subsubsection{memo}
\label{sec-32_1_5}

【大好きヤフー知恵袋】
ゴリラは床屋へ行くのですか?
金しだいとか
\subsubsection{\textbf{DONE} ヽ(゚∀゚)ノ 30回目}
\label{sec-32_1_6}

【大好きヤフー知恵袋】

食パンマンの顔は何枚切りなんですか?

何枚切りだろうが、彼は二枚目です。

(これは秀逸でした
\subsection{MIRACLE}
\label{sec-32_2}

anna@bayfm.co.jp

アンナさん! おはようございます。

テーマ「鍋」とのこと。
昨夜鍋だったのであまり面白くないかもしれませんが、その内容をお話しします。

朝一人で出掛けたので帰りに鍋の材料を買い込みました「今夜は鍋!」と決めて。

夜、土鍋へ豚肉、白菜、ねぎ、エリンギ、エノキ、きぬごし豆腐 を投入して煮込み食卓へ。

5歳2歳の子どもたちは少し食べたら「うどん!」となり、ろくに僕は食べられないままそちらの用意へ。
カセットコンロなど無いのでまたキッチンに戻って、うどんを一玉投入してぐつぐつ。
ほとんどチチハハの手がつかないまま、おうどんはチビたちのお腹の中に啜られていきました。

やっと座れて鍋をつつき始めたら今度はヨメサンの方が「雑炊食べたい」

仕方なくまたキッチンへ。
冷やご飯を軽く水洗いして投入、塩をひとつまみ入れ、ヨメサンは柔らかめが好きなので時間長めにグツグツ。
小ネギを刻んで入れ軽く交ぜ、溶いた玉子を回しかけその上に刻んだ水菜を振りかけて蓋をして火を止める。

ヨメサンはパクパク食べてくれました。
慌しいのはいつものことなので、みんなに満足してもらえて僕も満足という夜でした。美味しかった。
\subsubsection{\textbf{TODO}}
\label{sec-32_2_1}

【自己覚醒スペシャル】
\section{2011-11-22}
\label{sec-33}
\subsection{女子力うぜぇ // 暗い人祭り}
\label{sec-33_1}
\subsubsection{memo}
\label{sec-33_1_1}

【女子力うぜぇ】

女性っぽく見せること?
\subsubsection{\textbf{DONE}}
\label{sec-33_1_2}

【女子力うぜぇ】

おじさんはとりあえず「女子力」とはなんぞや、とネットで検索しようとしたら
指が滑って「女子寮」で検索してしまいました。

そちらの方の検索結果の確認に忙しかったため。いまだ分かっておりません。

「女子力」って何ですか?
\subsubsection{\textbf{DONE}}
\label{sec-33_1_3}

「女子力」。
いくらネット検索してもピンとした答えが見つからず、ヨメサンにメールで訊いてみました。

妻からの返信:
「泣き叫んでいる我が子を片手で抱っこしながら、もう片方の手で重く詰まった買い物袋を持って歩ける。」

キミか!キミが女子力か!

\ldots{}違うそれは「かあちゃん力」だ。
\subsubsection{\textbf{DONE} ヽ(゚∀゚)ノ 28回目}
\label{sec-33_1_4}

【暗い人祭り】

スイカの食べかた
\subsubsection{\textbf{DONE} ヽ(゚∀゚)ノ 29回目}
\label{sec-33_1_5}

パンツを切り刻む
\section{2011-11-21}
\label{sec-34}
\subsection{外国人にはよーく説明しないとわからないもの // いらないものランキング}
\label{sec-34_1}
\subsubsection{memo}
\label{sec-34_1_1}

【外国人にはよーく説明しないとわからないもの】

愚息
愚妻

【いらないものランキング】
クーピーの金。

指輪を買うと付いてくる小さい袋。

100ページ以上あるコンピュータの使い方マニュアル。
オンラインにしてくれよ。

スーツのポケットにいつまでも入っている、布キレ。

ピス兄さんに整髪料。

やたら勢いの無い、公衆トイレの「ジェットタオル」
エコかもしれないけど、いたずらに電力使っているだけの気がしてならない。
\subsubsection{\textbf{DONE}}
\label{sec-34_1_2}

【外国人にはよーく説明しないとわからないもの】

引越しの餅投げ。
\subsubsection{\textbf{DONE}}
\label{sec-34_1_3}

時代劇の中には説明し難いものがたくさんありそうですね。

桜吹雪とか
印籠とか。水戸黄門とか。
「What's KOMON?」とか云われそう。
\subsubsection{\textbf{DONE}}
\label{sec-34_1_4}

サッカー好きの外人さんに「やまとなでしこ」の本当の意味を説明できるか。
\subsubsection{\textbf{DONE}}
\label{sec-34_1_5}

どっちがおすぎで、どっちがピーコか
\subsubsection{\textbf{DONE}}
\label{sec-34_1_6}

集めた落葉で焚き火をして焼き芋を焼く行為。
\subsubsection{\textbf{DONE}}
\label{sec-34_1_7}

天狗のお面 を使った AV。

女体盛りの意義。

餅喰い叔父さん。
\subsubsection{\textbf{TODO}}
\label{sec-34_1_8}

【いらないものランキング】
\section{2011-11-18}
\label{sec-35}
\subsection{KIRAKIRA お米と私}
\label{sec-35_1}
\subsubsection{\textbf{DONE}}
\label{sec-35_1_1}

「お米と私」

ほぼ毎日土鍋でご飯を炊いてます。
しかし、そろそろその土鍋を卒業しなければならなくなりそうなのです。

ヨメサンと同棲してた頃は大人とはいえ二人だけ。
仕事をしているので3食家でということもなく、片手鍋でご飯を炊いていました。

ちなみに炊飯ジャーは何年も使っていないです。
さらにちなみに、家にいる時は僕が食事を作るので、朝、ご飯を炊くのはほとんど自分です。

結婚して、長男が生まれ、次男が生まれ。
そしてこの前、朝に炊いた3合のご飯が朝食時点で尽きました。

「ついにこの日が来たか!」と戦慄しました(笑)

5歳の長男は「ああ、食べすぎたお腹いたい」と云っていたので、まだもうちょっと大丈夫そうですが。

使っている土鍋は3合炊き。ついにジャーを買うかと悩んでおります。
嬉しいヒメイってヤツなのですが。

まぁヨメサンの実家が農家なのでお米は買ったことが無いですこの10年、ありがたや。
\subsection{ハピリー}
\label{sec-35_2}

tama@joqr.net
\section{2011-11-17}
\label{sec-36}
\subsection{学習能力まるでなし // 見られたくない場面ランキング}
\label{sec-36_1}

ブクマ二度する
同じコミックを二つ三つ買う
\subsubsection{\textbf{DONE}}
\label{sec-36_1_1}

【学習能力まるでなし】

この前やってしまいました、スライサーで親指の爪の端っこを削る。
子どもの頃から何度もやっているのにいまだにときどきやります。

そのときはまあ、酔っぱらいながら料理してたのが問題かもしれません。
\subsubsection{\textbf{DONE}}
\label{sec-36_1_2}

飲み会のたびに同じことを毎回訊かれると、ああ自分に興味ないんだな、と分かります。
自分も他の人にやってないか心配になりますね。
\subsubsection{\textbf{DONE}}
\label{sec-36_1_3}

同じ漫画を二つ買う、というところまでは許容範囲でしたが。
ついに三つ目を買ったときには愕然としました。
\subsubsection{\textbf{DONE}}
\label{sec-36_1_4}

酔った帰り途に電車を乗り過ごし、最終電車も終っていて2時間くらい歩いて帰る。
ということを数回立て続けにやってしまったことがありました。

そして深夜遅くに帰り、朝遅くに起きて思い出したのは、前回やっちまったとき
「そういうときはカードを使っていいからタクシーで帰ってきなさい」
というヨメサンのお言葉。

「学習能力がない」というより、たぶん酔っぱらって歩いて帰るのが好きなんだと思います。老化じゃなく
\subsubsection{\textbf{DONE} ヽ(゚∀゚)ノ 26回目}
\label{sec-36_1_5}

目薬を眼鏡の上からさす
\subsubsection{\textbf{TODO}}
\label{sec-36_1_6}

【見られたくない場面ランキング】
みみげを
\subsubsection{\textbf{DONE} ヽ(゚∀゚)ノ 27回目}
\label{sec-36_1_7}

Tシャツ嗅いでる
\subsection{サラドレ}
\label{sec-36_2}

浮気のラインは。黙って異性と会う、かなぁ。ヨメサン嘘下手だから、きっと分かっちゃうからしないでね
\subsection{KIRAKIRA}
\label{sec-36_3}

ラジオネーム: りとるぐれい (神奈川県川崎市在住40才男)

「もしもしカメです 亀さんです」

小島さん、ピエールさん、こんにちは。

亀でいうと、ガメラの手足がひっこんで、そこから火を吹いて、さらに回転して飛ぶ、
というシステムがどうしても納得できません。

瀧さんなりの回答をいただけますでしょうか。
\section{2011-11-16}
\label{sec-37}
\subsection{GLZ カチンとくる しゃべり教室 // うすうすランキング}
\label{sec-37_1}
\subsubsection{\textbf{TODO}}
\label{sec-37_1_1}

【カチンとくる しゃべり教室】

内容/仕草とか

おたくっぽい
目を合わせない
目が自分の上の方を泳いでいる

内容はなんでもいいが、視線が自分の生え際あたりをさまよう
ピス兄さんみたいに生え際無くしちゃえ

美味しいは一言も云わず、文句だけ云うタイプ

たぶんすごくイイヤツなんだと思うけど、杉山ハリー

お弁当男子。
\subsubsection{\textbf{DONE} ヽ(゚∀゚)ノ 25回目}
\label{sec-37_1_2}

先輩って、床屋行くんですか?
\subsubsection{\textbf{DONE}}
\label{sec-37_1_3}

うすいもの(概念)

【うすうすランキング】
ちょっとテーマとは違うと思うのですが、「うすうす」で思い出したこと。

子どもの頃親のタンスを物色してたときに見つけた
「うすうす」
コンドームでした、軽くトラウマです。

大きくなって装着してみましたが普通でした。
\subsubsection{\textbf{TODO}}
\label{sec-37_1_4}

AV女優になったきっかけ「歌手になりたかった」
\section{2011-11-15}
\label{sec-38}
\subsection{GLZ ふびん祭り // 不安ランキング}
\label{sec-38_1}
\subsubsection{\textbf{DONE}}
\label{sec-38_1_1}

【ふびん祭り】

はげ
残されたサンダル
\subsubsection{\textbf{DONE} ヽ(゚∀゚)ノ 23回目}
\label{sec-38_1_2}

長く残った白いクーピー
\subsubsection{\textbf{DONE} ヽ(゚∀゚)ノ 24回目}
\label{sec-38_1_3}

アイコラの、体の方の人。
\subsubsection{\textbf{TODO}}
\label{sec-38_1_4}

【不安ランキング】

車のライトを消したっけ?

コンが破れてる!
「あれが来ないの」と云われる

ピン 暗い店内でお嬢が来たが、どこかで会ったような、、、知り合いのような気がしてならない。
が、コトが始まる。

毛根が閉じてきた!!

枕の抜け毛が力ない
\section{2011-11-14}
\label{sec-39}
\subsection{GLZ するするスルー // オッサン先祖がえり}
\label{sec-39_1}
\subsubsection{\textbf{TODO}}
\label{sec-39_1_1}

【するするスルー】
\subsubsection{\textbf{DONE}}
\label{sec-39_1_2}

【オッサン行動ランキング】

耳毛を抜く。

髪以外の毛に白髪を見つけてショックを受ける。

風呂あがりは全裸、あまつさえ、全裸の子どもたちと「裸族」と称して踊る。
\subsubsection{\textbf{TODO}}
\label{sec-39_1_3}

社内の飲み会のとき。
女の子にエロトークを振るが反応が悪く逆ギレ。次の日バツが悪くて休む。

いましたそういうヒト。
\section{2011-11-10}
\label{sec-40}
\subsection{サラドレ 結婚とは?}
\label{sec-40_1}
\subsubsection{\textbf{DONE}}
\label{sec-40_1_1}

結婚とは、自分のホームを作ること。
そこは安住の地であり、忍耐の場であり、もっとも幸せの感じられる場所であります。
\subsubsection{\textbf{DONE} ヽ(゚∀゚)ノ 2回目}
\label{sec-40_1_2}

(自分のポストじゃなくて、)

結婚とは「通過点である」という方がいましたが、
それは「ゴールではない」ということでしょう、
結婚は終わりじゃなくて、始まりだよ、ということと、受け取り共感しました。

フォローしたのが読まれたw
\subsection{GLZ 子供の頃の夢と今の私//ブログクイズ}
\label{sec-40_2}
\subsubsection{\textbf{DONE}}
\label{sec-40_2_1}

【子供の頃の夢と今の私】

実家の子どもの頃のアルバムを見ると、なんやら叫びながら塀から飛び降りている自分がいます。
キャプションに「仮面ライダー とう!!」と母親の字で書いてあります。

仮面ライダーになりたかったようです、あんまり覚えてないのですけど。

さいきん5歳の長男に「将来なりたいもの」を訊くと

 『仮面ライダー』。

僕の夢は子どもに引き継がれました。

でも、さいきんはライーダになりたいならイケメン俳優にならないとダメだよ。
\subsubsection{\textbf{TODO}}
\label{sec-40_2_2}

作家
先生
ゴレンジャー
たんてい
どろぼう
\subsubsection{\textbf{TODO}}
\label{sec-40_2_3}

子どもの時の夢ということで記憶を検索しているのですが、大したものが出てきませんでした。

たぶん子どもの頃に思っていたより、いまは良い生活をしているような気がします。

ひねくれた子どもだったんでしょう。
\subsubsection{\textbf{TODO}}
\label{sec-40_2_4}
\section{2011-11-09}
\label{sec-41}
\subsection{過保護祭り // 日本好きの外国人}
\label{sec-41_1}
\subsubsection{\textbf{DONE}}
\label{sec-41_1_1}

【過保護祭り】

さいきんの運動会での話だったと思います、何かで読んだのですが。

徒競走のトラックに走るのが遅い子用の近道があるとか。

そんなことなら、運動会なんてしなくていいのにね、と思いました。
\subsubsection{\textbf{DONE} ヽ(゚∀゚)ノ 22回目}
\label{sec-41_1_2}

「過保護」かは分かりませんが、先日両親が家に泊まりに来たときのこと。

食事中に母親から「よく噛まないとだめですよ」と叱られました。
今年のこと、数ヶ月前のことです。

今年40歳になりました。
\subsubsection{\textbf{DONE} ヽ(゚∀゚)ノ 23回目}
\label{sec-41_1_3}

【日本好きの外国人】

その昔英会話教室に行っていた頃。
僕の名前が「アキラ」と分かると (キラーン) と目が輝き。

「オー、アキラ! ナイスネーム!」

握手を求められることがちょこちょこありました。

マンガ、映画の『AKIRA』が好きな外国人は多かったですね。
\section{2011-11-08}
\label{sec-42}
\subsection{自分への罰 // 世界のおいしいバイト}
\label{sec-42_1}
\subsubsection{\textbf{TODO}}
\label{sec-42_1_1}

【自分への罰】

XXすると、

たべない
ビール飲まない
おなきん
\section{2011-11-07}
\label{sec-43}
\subsection{テレビで知った事を披露する会 // モテない理由}
\label{sec-43_1}
\subsubsection{\textbf{TODO}}
\label{sec-43_1_1}

「テレビで知った事を披露する会」
\subsubsection{\textbf{DONE} ヽ(゚∀゚)ノ 21回目}
\label{sec-43_1_2}

モテない理由

頬肉にメガネが乗ってる。わらうとひくひく
\section{2011-11-04}
\label{sec-44}
\subsection{KIRAKIRA 開けたり、閉めたり、叩いたり 扉やドアの話}
\label{sec-44_1}
\subsection{\textbf{TODO}}
\label{sec-44_2}

「開けたり、閉めたり、叩いたり 扉やドアの話」

子どもと二人のとき、お母さんがベランダで洗濯物を欲している最中に子どもが部屋の鍵をかけてしまい、お母さんが閉め出されてしまう、ということがよくあります。

長男がまだ小さかった頃、妻がまさにこのように閉め出されてしまったことがあったそうです。

ただヨメサン、この手の話はマンション内の他の奥様がたに聞いていたので対処方法は分かっていたそうです

もし自分がこういったことになったら「何やってんだ!!」と普段しないような怖い顔をして、チビを泣かしてしまい、お隣さんの気配がするまでベランダで過ごすことになったでしょうね。

ヨメサン、とにかくにこやかにチビを鍵のところまで誘導して、開けてもらったそうです、「きっ」となって怒るとどうにもならなくなるのがオチだそうです。

ハタで見てみたい情景ですが、ハタで見てたら「なんとかしろ」と云われると思うのでなんかつまんない。
\section{2011-11-02}
\label{sec-45}
\subsection{ほめごろしライブ // やっちまったランキング}
\label{sec-45_1}
\subsubsection{\textbf{TODO}}
\label{sec-45_1_1}

「ほめごろしライブ」
\subsubsection{\textbf{TODO}}
\label{sec-45_1_2}

「やっちまったランキング」

ハロウィン扮装のまま

となりの車の
\section{2011-11-01}
\label{sec-46}
\subsection{GLZ 雑フェス // 屈辱ランキング}
\label{sec-46_1}
\subsubsection{\textbf{TODO}}
\label{sec-46_1_1}

「雑フェス」

コンビニ
病院
商品
ラーメン屋

昔話の結末
対応
\subsubsection{\textbf{TODO}}
\label{sec-46_1_2}

「雑フェス」
\subsection{KIRAKIRA}
\label{sec-46_2}
\subsubsection{\textbf{DONE}}
\label{sec-46_2_1}

ラジオネーム: りとるぐれい (神奈川県川崎市在住40才男)

小島さん、堀井さん、こんにちは

「文化祭 エピソードまつり」

文化祭で思い出すことといえば、工業高校でのキャンプファイヤーですね。
工業高校なので男ばかりなのですが、なぜかキャンプファイヤーがありました。

校内のヒエラルキーの上の方の先輩がやぐらの上に立ち
「回れ! 回れ!」と叫ぶので、
火の周りをぐるぐると走って回ります。

「逆だ! 逆!」と今度は云うので、
反対回りに移行します。
洗濯機じゃあるまいし。。。

ただ走っているうちに、なにかだんだん気持ちよくなって来たのを覚えています。

もう少し青春な思い出が欲しかった。
\section{2011-10-31}
\label{sec-47}
\subsection{気取ってんじゃねえよナイト // ペットショップ西沢}
\label{sec-47_1}
\subsubsection{\textbf{TODO}}
\label{sec-47_1_1}

「気取ってんじゃねえよナイト」
\subsubsection{\textbf{TODO}}
\label{sec-47_1_2}

「ペットショップ西沢」

くちごたえしない
昨日どこいってたの? とか

云わない
\section{2011-10-28}
\label{sec-48}
\subsection{}
\section{2011-10-27}
\label{sec-49}
\subsection{マニアの醍醐味 // GLZ人探し}
\label{sec-49_1}
\subsubsection{\textbf{TODO}}
\label{sec-49_1_1}

「マニアの醍醐味」
\subsubsection{\textbf{DONE} ヽ(゚∀゚)ノ 20回目}
\label{sec-49_1_2}

「GLZ人探し」

彼女の家で浮気中、彼氏が帰ってきて、窓から逃げたことがある人。
あるいは、クローゼットに靴を持って隠れた人。
いますかー?
\section{2011-10-26}
\label{sec-50}
\subsection{昭和の精神論 // GLZおっさん知恵袋}
\label{sec-50_1}
\subsubsection{\textbf{TODO}}
\label{sec-50_1_1}

「昭和の精神論」
\section{2011-10-25}
\label{sec-51}
\subsection{忘れられない一言 // 軽く傷つくランキング}
\label{sec-51_1}
\subsubsection{\textbf{DONE}}
\label{sec-51_1_1}

「忘れられない一言」

エロいお店に行ったとき。

ことの最中に
「ネギ食べたでしょ」
と云われた、

萎えた。
\subsubsection{\textbf{DONE}}
\label{sec-51_1_2}


結婚したとき無職でした。

それ以前の会社は心身ともに疲れてしまって退職したのですが、
当然なんの計画もなく辞めたために次の仕事がなかなか見つからず。

結婚当日はプー太郎。
そんなプー太郎と結婚したヨメサンは後日云いました。

「詐欺かと思った」
\subsubsection{\textbf{DONE} ヽ(゚∀゚)ノ 19回目}
\label{sec-51_1_3}

軽く傷つくランキング 

\begin{enumerate}
\item コンビニ オデコのあたりを見る
\end{enumerate}
\section{2011-10-24}
\label{sec-52}
\subsection{小市民の小さな願い // 謎ランキング}
\label{sec-52_1}
\subsubsection{\textbf{TODO}}
\label{sec-52_1_1}

「小市民の小さな願い」

「こちらのどこからでも切れます」と書いてあるドレッシングの小袋。
ホントにどこからでも切れるようにして。
\section{2011-10-20}
\label{sec-53}
\subsection{GLZ自虐自分クイズ // だまされてそうランキング}
\label{sec-53_1}
\subsubsection{\textbf{TODO}}
\label{sec-53_1_1}

だまされてそうランキング
エロビ パッケージ
雑誌表紙
ちち
\subsection{KIRAKIRA あの時、君はどうかしていた イタ~い過去告白スペシャル}
\label{sec-53_2}
\subsubsection{\textbf{DONE}}
\label{sec-53_2_1}

ラジオネーム: りとるぐれい (神奈川県川崎市在住40才男)

「あの時、君はどうかしていた イタ~い過去告白スペシャル」

小島さん、瀧さん、宇多さん、こんにちは。

さきほど、「アロンアルファで完全体に」というお話がありましたが。

それを聞いて、
おもいだしたああああああ!!!!

小学校低学年の頃、男子たちの間では、まだ恥かしげにしているソイツを
完全体にしたり、戻したり、する遊びがはやっていました。

僕はどうも伸びが良くなくて、がんばって訓練しました。

そして、ついにソイツは進化を遂げました。
「やったー!!」と思ったら、今度は戻りません。

どうやっても戻らないんです、あんなに恥かしげにしていたのに急に
主張が激しくなったようでした。

そしてその状態のままでは痛いんです。
半日ほどがんばってなんとかやり過ごしましたが、けっきょくは親に告白して
病院に担ぎこまれましたとさ。

あ、いまは、無事完全体です。
\section{2011-10-19}
\label{sec-54}
\subsection{ヒーローのヒミツ // これこそチェリーだ!ランキング}
\label{sec-54_1}

ヒーローの秘密
キティちゃん、おすぎとピーコ、おおやまのぶよ、ゆりげらー もとふぁっしょんもでる
\subsubsection{\textbf{TODO}}
\label{sec-54_1_1}

「ヒーローの秘密」
\subsubsection{\textbf{TODO}}
\label{sec-54_1_2}

「これこそチェリーだ!ランキング」

意味もなく河原をうろつく。
\section{2011-10-18}
\label{sec-55}
\subsection{一言劇場~ウソつき編~ // 勇者ランキング}
\label{sec-55_1}
\subsubsection{\textbf{DONE}}
\label{sec-55_1_1}

「一言劇場~ウソつき編~」

今夜、マツコデラックスとベッドインします。

(よくわかんないけどこんなのですか)
\subsubsection{\textbf{DONE}}
\label{sec-55_1_2}

勇者ランキング

勇者と云えばライディーンです。
\section{2011-10-17}
\label{sec-56}
\subsection{平成迷い道//どうでもいい知識ランキング}
\label{sec-56_1}
\subsubsection{\textbf{DONE}}
\label{sec-56_1_1}

「平成迷い道」

iPhone4S 買うかどうかではなく、
メモリ 32GB にするか 64GB にするか。

月上乗せ 400円で 64GB にしました。
\subsubsection{\textbf{DONE}}
\label{sec-56_1_2}

「どうでもいい知識ランキング」

爪楊枝の指でつまむところにぐるっと窪みがあるじゃないですか何本か。

そこをポキリと折り、箸置きのように『爪楊枝置き』にして、
知った顔をして「こうやって使うんだよ」て云うオヤジがいるじゃないですか。

あれはただの飾りです、メーカーの人が云ってました。
\subsubsection{\textbf{DONE} ヽ(゚∀゚)ノ 18回目}
\label{sec-56_1_3}

こなきじじい 2トン
\section{2011-10-13}
\label{sec-57}
\subsection{ドブ金レポート//神、降臨。全ての疑問に答えます!}
\label{sec-57_1}
\subsubsection{\textbf{DONE}}
\label{sec-57_1_1}

「ドブ金レポート」

東京電力株を少量ながら持っております。

まさかここまで事態が悪くなるとは思っておりませんでした。
完全塩漬け状態です。

「ドブ金」になるかどうかはまだ分かんないんですけどねー
\subsubsection{}

英語教室

漫画

習いごと
\subsubsection{\textbf{DONE}}
\label{sec-57_1_3}

「神、降臨。全ての疑問に答えます!」

爪楊枝の持つところにある、あの模様はなんですか?
毎月お金が足りません


来年ピストン西沢の頭に生えるという木の名前を教えてください

エレベータの「閉まるボタン」をあるリズムで押すと別の世界に連れて行かれると云いますが、一体どこの世界に行くのでしょう?
\subsubsection{\textbf{DONE} ヽ(゚∀゚)ノ 17回目}
\label{sec-57_1_4}

焼きそばのお湯をシンクに捨てると「ボコッ」と音がしますが、一体誰が下から棒で突き上げているのですか?
\section{2011-10-12}
\label{sec-58}
\subsection{外国人に憧れる日本人が聴いたらいい番組 // 切ないランキング}
\label{sec-58_1}
\subsubsection{\textbf{DONE} ヽ(゚∀゚)ノ 16回目}
\label{sec-58_1_1}

「切ないランキング」
チビの好物 ミートボールの残り汁ご飯
\section{2011-10-11}
\label{sec-59}
\subsection{完全匿名 人の秘密をもらす会 // カントリー575}
\label{sec-59_1}
\subsubsection{\textbf{DONE} ヽ(゚∀゚)ノ 15回目}
\label{sec-59_1_1}

「完全匿名 人の秘密をもらす会」

会社の共有ファイルサーバ。
温厚なSさんのディレクトリのをどうしてか見てしまった

会社の人、共有ファイルサーバにエロ。
\subsubsection{\textbf{TODO}}
\label{sec-59_1_2}

まぁ僕はオッサン子持ちなので話してくれたんだと思うけど、
同僚の彼女持ちI君から猛烈プッシュがあるらしい

「気持ち悪いとか云うなよ、夕食くらい付き合ってあげてよ」と云うのが精一杯でした。
\subsubsection{\textbf{TODO}}
\label{sec-59_1_3}

「カントリー575」

蜂の子に 蚕のサナギ 大好きだ

熊が出た! 春の息吹きを 感じます

ゲロゲロォ カエルかな? ウウン酔っぱらいオヤジ
(字余り)

カッパかな 

決闘だ!! おばあちゃんと 野良猫が

タロとジロ スイカ泥棒 捕まえた

===


半袖半ズボン

デキ婚の 同級生二人 子どもハタチ

ひぐま

単線乗り遅れると 一時間後

小学校 新入生二人 

見逃すと 単線 電車
\section{2011-10-06}
\label{sec-60}
\subsection{KIRAKIRA 池・沼・湖について存分に語るがいい}
\label{sec-60_1}
\subsubsection{\textbf{DONE}}
\label{sec-60_1_1}

さて、「池・沼・湖について存分に語るがいい」とのことですが、

実家の道から面して後ろ側に25mプールくらいの大きさの沼があります。
貯水池らしく、そこで「遊び」をすると小学校にクレームが入りました。
ゴムボート浮かべてた先輩は呼び出しを受けてました。釣りもダメ、基本近づくなという感じ。

幼稚園くらいの頃、夜によくその沼に連れていかれました。

母親が本気で怒ったときに僕を抱っこして、沼の淵から暗い水面を見下ろしながら、淡々と小言を云われたのを思い出します。

「これ、ぼく、落とされるのかなぁ?」と本気で震えましたね。
そのときの恐怖は覚えているのですが、何について怒られたのかは一切覚えていません。

ので、躾としては間違ってますよね。トラウマですよ完全に。
母親を小一時間問いつめたい感じです。

母の躾の最終兵器「後ろの沼」今も実家に帰ると変わらずそこに水をたたえてあります。
見ると一瞬で子どもに戻ります。
\subsubsection{\textbf{DONE}}
\label{sec-60_1_2}

「池・沼・湖について存分に語るがいい」

上から目線はどうしてなのか分かりませんが、

実家の道から面して後ろ側に25mプールくらいの大きさの沼があります。

子どもの頃のこと、冬凍った水面の上にうっすら雪が積もっているその池を見ました。
雪がちらちらとしていたかと思います。

よく見るとその上に足跡が点々とあることに気付きました。

自分いる淵の反対側からこちらに向かって足跡が続いてきます。
その足跡は池の真ん中あたりで止まっていました。

「危ないなぁ、氷が割れたらどうすんだよ」と子どもながらに思いました。
その貯水池はすり鉢状になっているので誤って落ちたら大変なことになります。

そして変なことに気付きました。
真ん中へ向って行く足跡はありますが、返って行く足跡が無いんです。
てっきりそこで折り返して同じ道を戻ったものと思っていたのですが、目をこらしてもありません戻って行く跡が。

なんだか怖くなって家に戻りました。
だれかのいたずらだったのでしょうか。。。
\subsection{GLZ うすいセリフ // 忙しい人にかまってもらう会}
\label{sec-60_2}
\subsubsection{\textbf{DONE}}
\label{sec-60_2_1}

「あたしのこと、どのくらいスキ?」

「んー、このくらいかな」

人差し指と親指にすき間を作る男。

「えー、そんなにチョットだけー」

「ただ、地球はこんだけだけどな」

指の間のすき間を短かくして示す。

「きゃー」(バカ)
\subsubsection{\textbf{DONE}}
\label{sec-60_2_2}

「よーし、いい子だ」
\subsubsection{\textbf{TODO}}
\label{sec-60_2_3}

心のない言葉だ。
\section{2011-10-05}
\label{sec-61}
\subsection{KIRAKIRA 勝手にやっちゃった 勝手にやられちゃった}
\label{sec-61_1}
\subsubsection{\textbf{TODO}}
\label{sec-61_1_1}
\subsection{東京スーパーローカル情報 // おっさんチンプンカンプンクイズ}
\label{sec-61_2}
\subsubsection{\textbf{TODO}}
\label{sec-61_2_1}

【東京スーパーローカル情報】

上井草駅にガンダムがあるように

わが地元駅に「ドラえもん」のモニュメントができました。

ただ問題は東京じゃないので今回のテーマとちょと違う。
\subsubsection{\textbf{DONE} ヽ(゚∀゚)ノ 14回目}
\label{sec-61_2_2}

【東京スーパーローカル情報】

コンビニにーちゃん、くるくるして
食いこむ
\section{2011-10-04}
\label{sec-62}
\subsection{KIRAKIRA 「地名とか住所のお話」}
\label{sec-62_1}
\subsubsection{\textbf{DONE} ヽ(゚∀゚)ノ !!!! 初めて}
\label{sec-62_1_1}

「地名とか住所のお話」

小島さん、堀井さん、こんにちは。

うちの実家近くに「だしな (駄科)」というところがあります。
長野県飯田市です。

その駄科駅に電車が止まると、

  「だしなー、だしなー」

軽く毎回カツアゲされてる感があります。
\subsection{GLZ 最近言わなくなったね // 秋のジャイアン祭り}
\label{sec-62_2}
\subsubsection{\textbf{DONE}}
\label{sec-62_2_1}

【GLZ 最近言わなくなったね】

チャンネルを「回す」。
云ってしまった後、子どもたには分からないよなとか思います
\subsubsection{\textbf{DONE}}
\label{sec-62_2_2}

「巻き戻し」。
カセットテープ、ビデオテープ、DAT
などシーケンシャルアクセスするものは無くなりましたねー。
でも言葉は使うのか。。。
***
愛してるって最近言わなくなったのはー♪
イエ、けっこう云ってます。
\subsubsection{\textbf{DONE}}
\label{sec-62_2_3}

「アンテナ」
ラジオ、テレビ、携帯などで、使用前に手で伸ばすようなアンテナは無くなりましたね。
語源である虫の触覚のような形状のものは、とんと見掛けなくなりました、みんな内蔵に変わってしまったのでしょうか。
いま分かりやすく「アンテナ」しているのは、ラジコンのコントローラくらいでしょうか。
それさえ内蔵型になっているんですかね。
\subsubsection{\textbf{TODO}}
\label{sec-62_2_4}

まぐわい
\subsubsection{\textbf{DONE}}
\label{sec-62_2_5}

ビフテキ
\subsubsection{\textbf{DONE} ヽ(゚∀゚)ノ 12回目}
\label{sec-62_2_6}

さいきん云わなくなったこと。

ちょっとタンマタンマ
\subsubsection{\textbf{DONE} ヽ(゚∀゚)ノ 13回目}
\label{sec-62_2_7}

鼻血ぶー
\section{2011-10-03}
\label{sec-63}
\subsection{GLZ あまり使わないもの祭り}
\label{sec-63_1}
\subsubsection{\textbf{DONE}}
\label{sec-63_1_1}

あまり使わないもの祭り

・マウスの真ん中ボタン

・靴べら

・色鉛筆の金銀白

・刺身のツマ。食べない


・カレー屋の割引券。期限が切れて(あああ)てよくなる

・育毛剤。もう諦めてる
\subsubsection{\textbf{DONE} ヽ(゚∀゚)ノ ひさびさ 11回目}
\label{sec-63_1_2}

あまり使わないもの祭り
イヤフォンについてくる、小さい袋。
\subsubsection{\textbf{DONE}}
\label{sec-63_1_3}

GLZ流 生活の知恵

フローリングに米のとぎ汁を使って拭くとピカピカになります。
\subsubsection{\textbf{DONE}}
\label{sec-63_1_4}

iPadなどのタブレットPCにAVを入れておくと、

いろんなシチュエーションのバーチャルエロが楽しめるぞ!!!
\section{2011-09-29}
\label{sec-64}
\subsection{KIRAKIRA 思わず声が出ちゃった}
\label{sec-64_1}

***
思わず声が出ちゃった

数年前のこと。
電車で吊り革にぶらさがっていたら、小さな赤ちゃんを胸に抱えたママさんが乗って来ました。
すこし離れていたのではっきりとは分かりませんが、生まれて二ヶ月くらいの子でしょうか。
ママさんは赤ちゃんの扱いがまだ慣れていない感じ、自分もこんなだったなぁ、と我が子の生まれた頃を思い出しました。

内心 (座っている人代わってあげないのかな) と思ったら、そのすぐ前の人が立ち上がりました。

その場にいた人たちが和んだ空気になりました、僕含め。
ママさんその方にお礼を云ってから、 シートの方向に体を向けたのですが、
そのとき

 「コンっ」

赤ちゃんの小さな頭が、座席の端にある鉄の手すりに当たってしまいました。

「ああ」

面白いことに見てた人たちがみんな小さな声を上げました。
すぐに泣き出す赤ちゃん、あやすお母さん。

また和みました(笑)

そして、あの鉄柱を切り落したくなりました。
\subsection{GLZ 小学生の頃すごかったこと}
\label{sec-64_2}
\subsubsection{\textbf{DONE}}
\label{sec-64_2_1}

小学生の頃すごかったこと

庭の一角に線路が3メートルほどありました。
安全教室とかで使うものであったと思ったのですが。

そうすると子どもたちの間でどういうことになるかというと、

「ここに昔電車通ってたんだぜ」と云ってました。

子どもって、脳があれですよねぇ。
\subsubsection{\textbf{DONE}}
\label{sec-64_2_2}

友だちの誕生日会に初めて行ったときのことですが。

よく分からず、プレゼントにチョロQを買っていったら、他の友だちたちのプレゼントがすごかった。
プラモデルもおもちゃも高そうなものばかり。
掌に乗るようなものは僕だけでした。

かるくトラウマになりました。お誕生会こわいと。

友だちの中で一番親近感が湧いたのが、一時期はやったマーカーとそれを消すマーカーのセットのもの。
「これすげーんだぜ、遊ぼうぜ」って云ってた、ヒロシくん。
\subsubsection{\textbf{DONE} よまれたけど、ビミョーだw}
\label{sec-64_2_3}

初心者GLZ

りとるぐれ子

メールしても全然読まれませんね。
選ばれるコツってなんなのでしょうか?
もしなにかあるなら教えてくださいっ。
\section{2011-09-27}
\label{sec-65}
\subsection{男汁祭 // 重い男と女ランキング}
\label{sec-65_1}
\subsubsection{\textbf{DONE}}
\label{sec-65_1_1}

『男汁祭』

企業戦士は家に帰るとまず上半身ハダカになり。

iPhoneを裸の胸に装着。
腹筋トレーナーアプリを起動し、スタート!

「軍曹バージョン」にしてあるため掛け声は。

『フン・フン・イヤッ!!』
  『フン・フン・イヤッ!!』
     『フン・フン・イヤッ!!』

掛け声に合わせて腹筋。
ハラに集中、丹田に力を込める。

コンディションが良ければ10分で150回越え。

その頃には汗びっしょりになり、iPhoneの背中には40男の男汁がべったりと付着。

うーん、イイネ (`・ω・´)キリッ
\subsubsection{\textbf{DONE}}
\label{sec-65_1_2}

高校生の頃、自家発電後の男汁の処理をどうしているか、という話題になりました。
僕は普通にゴミ箱に捨ててたんですが、それを友人たちに云うとめちゃくちゃdisられました。
それからは心を入れかえ、ちゃんとトイレに捨てるようになりました。
\subsubsection{\textbf{DONE}}
\label{sec-65_1_3}

「なんだよ、今どき王冠の栓だよ。栓抜きないよ、どうすんだ?」

「\ldots{}おれに貸しな」

「ああ、でもピスどうすんだよ、栓抜き無いんだゼ」

(キラーン)

「え?」

(かぱーん!!)

「ほらよ」

「\ldots{}お前\ldots{} いい歯だな」
\subsubsection{\textbf{DONE}}
\label{sec-65_1_4}

「お父さんのオデコっていつも光ってるね」

男汁じゃい。ムスコよ。
\section{2011-09-26}
\label{sec-66}
\subsection{ネガティブ法則 // 自分ライフのすすめ}
\label{sec-66_1}
\subsubsection{\textbf{DONE}}
\label{sec-66_1_1}

【ネガティブ法則】

これは帰れなくなるかもー早めに脱出だー、と会社を出る、
絶妙なタイミングで電車が発車、
だが絶妙なタイミングで電車が停止、地元駅まで数駅
二時間電車で過ごす、
ハラが減って電車を降りる、
マン喫で読み始めたマンガが止まらなくなる、
そこを出たらすっかり電車は動いてる、、、

先週21日の自分です。
正解はみなさん知ってる通り、台風が通り過ぎるのを待ってから動き出す。
\section{2011-09-22}
\label{sec-67}
\subsection{GLZ 街で見かけた有名人 // へなちょこランキング}
\label{sec-67_1}
\section{2011-09-20}
\label{sec-68}
\subsection{GLZ コンプレックス大開放 // ブルーアンケート}
\label{sec-68_1}
\subsubsection{\textbf{TODO}}
\label{sec-68_1_1}

背、髪、学歴
\subsection{TOKYO MORNING RADIO}
\label{sec-68_2}
\subsubsection{\textbf{DONE}}
\label{sec-68_2_1}

おはようございます、
毎朝6時から朝食を作りながら聞かせていただいております。

昨日某局の放送にて、小島慶子さんが別所さんとラジオの対談をしたと話されていました。

そのときは、仕事中で、片耳で聞いてましたのでちょっと聴き逃してしまいました。
どちらに掲載されるのがお教えください。

ちなみに小島さんは「すごくかっこよかった!!」と何度も云ってました。
大絶賛でしたよー。
\section{2011-09-16}
\label{sec-69}
\subsection{CIRCUS}
\label{sec-69_1}
\subsubsection{\textbf{DONE}}
\label{sec-69_1_1}

渡部さん、こんにちは

昨日夕方 GROOVE LINE Z で話題になっていたのですが。

CIRCUSCIRCUS でリスナーさんに紹介されたお店を「自分のいきつけの店」と
他メディアで云っていたそうですね。

本当なんすかー。

本当であれば、リスナーさんにきちんと謝ってください(笑)

ピストンさんは「そんなこと云うなよ、渡部はいいやつなんだよー」
と半笑いでフォローされてました。
\section{2011-09-15}
\label{sec-70}
\subsection{GLZ 感動ビンボー // 納得いかないランキング}
\label{sec-70_1}
\subsection{KIRAKIRA 「唇について熱く語ろう」}
\label{sec-70_2}
\subsubsection{\textbf{TODO}}
\label{sec-70_2_1}

「唇について熱く語ろう」

ブラスバンド部でしたので、

高い音を出すときは、腹筋に力を入れて音を出すのですが、なかなか思うようにならないときには
ついつい唇にマウスピースを押しつけてしまいます。

コンクール前など集中して練習したあとなど、金管部隊はみんなクチビル真っ赤で、すこしめくれあがった感じになったことを思い出します。
\subsubsection{\textbf{DONE}}
\label{sec-70_2_2}

「唇について熱く語ろう」

ヨメサンと結婚前、彼女がよく家に遊びに来ては泊まって行ってた頃。

1K の僕の部屋のカレンダーは「井上和香」さんでした。
むろん水着です。

ヨメサンはサッパリした性格だったので、そういうのは気にしないことと思っていたのですが。
実際、初めて家に来たときに「なにこれ」とニヤニヤ顔で僕に問いかけただけで、別に外して欲しいとは云いませんでした。

さて、いよいよ彼女が僕の家に引越して来るということになりました。

ヨメサンのご両親とお姉さんがその引越しのお手伝いに来てくれるということなり。彼女は電話口で僕に云いました。

「あの、クチビルの人のカレンダー、まだ貼ってあるの?」

ええ、外しましたよ即効。

その後、たまにクローゼットから取り出して眺めていたのは内緒です。
\subsection{miracle}
\label{sec-70_3}
\subsubsection{\textbf{DONE} 出すのおそすぎ}
\label{sec-70_3_1}

「思わずフリーズした話」

夜遅くに家に帰りつきました。
ヨメサンもチビたちももちろん夢の中。

寝室を覗いてから、抜き足で廊下を進み、僕は自室にかばんを置きに入りました。
すると、机の上に封筒が。

表に「おとうさんへ」、つたない長男(5歳)の字でした。
(まぁ、あの子ったら)と少しじんわりしながら封筒を開けました。

「きょう けむしを みつけました」

しばらくその紙を見て固まっていましたが、

すこしして「そうですか…」とつぶやきました。
\subsection{恋する秋 サラドレ}
\label{sec-70_4}
\subsubsection{\textbf{DONE} ヽ(゚∀゚)ノ よまれた}
\label{sec-70_4_1}

やっぱマッキーでしょう、40男として共感できるのは。他のラインナップは恰好良すぎ。恋なんてかっこいいもんじゃない。
\subsection{SUPRISE 「feat.キング・オブ・丼(どん!!)」}
\label{sec-70_5}
\subsubsection{\textbf{DONE}}
\label{sec-70_5_1}

「feat.キング・オブ・丼(どん!!)」

うーん、カツ丼か牛丼かなぁ。
カツ丼はお店によってクオリティがまちまちなので、安定感があるのはやはり牛丼ですかね。

私、いつか絶対に食べてみたい丼がありまして。

それが人形町駅近くにある「玉ひで」。
ランチに行っても普通に並ぶという人気の親子丼を食べてみたいぃぃ。

もしかしたら自分の親子丼の概念が覆されるのではないかと期待しています。

いつか、きっと行きますよー、ゼッタイ!!
\section{2011-09-12}
\label{sec-71}
\subsection{さめる話 // プロフィールクイズ~ウィキペディア編}
\label{sec-71_1}
\subsubsection{\textbf{DONE}}
\label{sec-71_1_1}

「さめる話」

(話を聞いていて思い出しましたことがありました)

エロビの女の子に蒙古斑が残っていたとき。
\subsubsection{\textbf{TODO}}
\label{sec-71_1_2}

かわいい声のお気に入りDJのラジオを聞いていて、

「いったい、どんな顔をしてらっしゃるのだろう」

とネット検索してしまったとき。
\section{2011-09-08}
\label{sec-72}
\subsection{メンタルが強い人 // あやしいことランキング}
\label{sec-72_1}

『メンタルが強い人』
\section{2011-09-07}
\label{sec-73}
\subsection{ケッ!っていう話 // 芸能人いらない情報ランキング}
\label{sec-73_1}
\subsubsection{\textbf{DONE}}
\label{sec-73_1_1}

『ケッ!っていう話』

お笑い芸人のダイエット。

お笑い芸人の美肌エステ。

お笑い芸人のカラオケ番組。

いずれも意味がわかんない。

ただ「おかもとまりの水着」はアリです。
\subsubsection{\textbf{DONE}}
\label{sec-73_1_2}

「ええー、ビールないのぉ? 
  アタシ、発泡酒も第三とかビールとか、飲めないんだよねぇ
  まずいじゃん」

とかいうオンナ。
\subsubsection{\textbf{TODO}}
\label{sec-73_1_3}

「店長のきまぐれサラダ」

きまぐれ系のものを見ると、「ちゃんと作れ」と思う。
\section{2011-09-06}
\label{sec-74}
\subsection{使える言い訳大百科 // 女捨ててるランキング}
\label{sec-74_1}
\subsubsection{\textbf{TODO}}
\label{sec-74_1_1}

「女捨ててるランキング」

幼稚園のバス停にたまに子どもを送りに行くのですが、そのバス停のお母様がたがかなり両極端です。

女捨ててる、捨ててない人の差が。

ばっちりメイクとアクセサリ、カジュアルだけどまとまっているというスタイルのお母さん。

かと思うと、ウェストゴムのジャージでもないパンツとTシャツで、ノーメイクのお母さん。

たぶん離れて見るとよく分からない団体です。

ヨメサンに云わせると、ダンナとうまく行っていないといった話をする人が「オンナを捨ててる」といった印象を受けるらしいです。
\section{2011-09-05}
\label{sec-75}
\subsection{KIRAKIRA}
\label{sec-75_1}

「地下道を作って」という方がいらっしゃしましたので便乗です。

私の地元『向ヶ丘遊園駅』に地下道を早く作っていただきたい!!
構想は何度も聞くのですがまったく着手される様子もなく。

藤子ミュージアムができて、たくさんの方がいらっしゃることになると思いますが
駅の不便さに辟易することは請け合います。

北口、南口の移動がたいへんなんです。最短距離は駅構内のみ。
なんとかしてくれーい!!
\subsection{GLZ 番長保存会 // しょうもない雑学ランキング}
\label{sec-75_2}

***
【しょうもない雑学ランキング】

アンパンマンの主要キャラクター、「メロンパンナ」と、その姉「ロールパンナ」。
じつは先にジャムおじさんが作ったのはメロンパンナ。ロールパンナはメロンパンナの要請により作られました。

アトムの両親と一緒ですね。
\section{2011-08-30}
\label{sec-76}
\subsection{今年の夏もあと一日 // 小心者ランキング}
\label{sec-76_1}
\subsubsection{\textbf{TODO}}
\label{sec-76_1_1}

「今年の夏もあと一日」
\subsubsection{\textbf{DONE}}
\label{sec-76_1_2}

「小心者ランキング」

飲み会のとき席を移動できません。
同席の人に「俺と話すの嫌なのかよ」と思われるとイヤだな、と。

トイレで用を足してるとき、声をかけられると、止まります。

自動改札通るとき。

チカンと間違われるといやなので、手はいつも塞がってます。
カバンとか吊革とか、ケータイとか。
僕は無実です。

書店などで、終了の音楽が流れると速攻帰ります。

時間チャージが加算されていくお店は料金が気になって一切楽しめません。
他人会計なら、いつまででもオッケーです。

健康診断の結果が届いても一ヶ月くらい放置。

コンビニでトイレを借りたら、なんか買わなくちゃと思います。
そして買います、普段は食べないガムとか。

投稿後、番組が進むにつれ「やっぱあのネタはねぇわ」となり、
読まれたらどうしようとクヨクヨします。
\section{2011-08-29}
\label{sec-77}
\subsection{ワイルド自慢 // こんな時どうする?アンケート}
\label{sec-77_1}
\subsubsection{\textbf{DONE}}
\label{sec-77_1_1}

「ワイルド自慢」

日常的に料理をします。家にいるときは基本料理担当です。

鍋、フライパンが温まったかどうかは、実際触って指先で確認するのが習慣になってまして。
沸騰しているお湯に指をつっこむとかも、けっこう平気な作業です。

以前パーティの仕込み中、家庭用のコンロで使えるダッチオーブンをプレヒートしてたとき。
いろいろと同時進行していたので頭が回っていなかったのでしょう、つい、いつもの調子で触りました。

『ジュッ』って云いましたよ、ワイルドでしょ。
\subsubsection{\textbf{DONE}}
\label{sec-77_1_2}

以前、iPhoneのケースに歯型がついていました。
下のチビがかじったようです。
正面に貼ってある液晶フィルムにも横一文字に傷がついてます。
当時生えたばかりの歯あとと思ってそのままにしてあります。

小さい子どもってのはワイルドなもんです。
\subsubsection{\textbf{DONE}}
\label{sec-77_1_3}

田舎が山の中なのでワイルドな食べものが好きです。

イナゴ、蜂の子、ざざむしなど。
酒のツマミに合います。

中でも一番好きなのが、蚕のサナギの佃煮です。
これあったら、ご飯何杯でも行けます。
***
名前だけ読まれた |ω・)
\section{2011-08-25}
\label{sec-78}
\subsection{ちょいM祭り// OH MY GOD キャンペーン}
\label{sec-78_1}
\subsubsection{\textbf{TODO}}
\label{sec-78_1_1}

イジるよりイジられたいネーム りとるぐれい

「ちょいM祭り」

ヒゲ剃りあとローションをつける。
傷口に刺激物を塗ってる感が好き。


サウナ
\section{2011-08-24}
\label{sec-79}
\subsection{人の家でびっくり!// 芸能人何でもアンケート}
\label{sec-79_1}
\subsubsection{\textbf{DONE}}
\label{sec-79_1_1}

【人の家でびっくり!】

ヨメサン筋の親戚の家に初めてお邪魔したとき。

焼酎のロックをいただきました。その氷が溶け出すときにシュワシュワと発泡するのが不思議だったのですが。

なんと、南極の氷とのことでした。

どういう経路でもらったと云っていたのかは忘れてしまいましたが。
発泡しているのは南極の空気か、とたいへん不思議な経験をさせてもらいました。
\subsubsection{\textbf{DONE}}
\label{sec-79_1_2}

「人の家」ではないのですが、ヨメサンの実家へ行ったとき。
6年前の引越し時に預けていた本入りダンボール10個ほどが無くなっていた。
訊くと全て家の本棚に並べたとのこと。

ヨメサンに隠して入れておいた、お気に入りのエロ本数冊がどこに行ったのかは定かではありません。
訊けません。。。
\subsubsection{\textbf{TODO}}
\label{sec-79_1_3}

小学生高学年くらいの頃、友だちの家に遊びに来てたときのこと。

突然ドアが開いて、
入ってきたのは友だちのお母さん。
僕は内心パニックになっていたのだけど、お母さんは普通に体を洗い始め、あたりさわりのない会話を始めました。
パニックを隠しながら返事をしていたのですが、気付いたら友だちの部屋に戻っていました。どうやって戻ったのかよく覚えていません。

なぜ昼すぎの時間に、友だちのお母さんが風呂に入ろうと思ったのかは、まったく分かりません。
\subsubsection{\textbf{TODO}}
\label{sec-79_1_4}

友だち夫妻の家に行ったとき、
ビデオクリーナーとして売っている、違法ドラッグがパソコンの上に乗っているのを見たときは軽く引きました。
\section{2011-08-23}
\label{sec-80}
\subsection{自虐自分クイズ // 歌詞イメージアンケート}
\label{sec-80_1}
\subsubsection{\textbf{TODO}}
\label{sec-80_1_1}

「歌詞イメージアンケート」

森高:
酔っぱらって作った慣れない手料理
\section{2011-08-22}
\label{sec-81}
\subsection{マヌケな瞬間100連発 // 芸能人いらない情報コンテスト}
\label{sec-81_1}
\subsubsection{\textbf{DONE}}
\label{sec-81_1_1}

「マヌケな瞬間100連発」

いねむりから覚めて
「あれ、ここ降りる駅じゃね?」となり

満員電車の人たちをかき分けて出口へ向かう。
(なんだコイツ) という顔をしつつもドアを開けてくれる人々。
目の前でドアが閉まり始める、、

間に合うかっ!? といちかばちかで突進する

なんとかホームへ降り立つ。

違う駅。
\subsubsection{\textbf{DONE}}
\label{sec-81_1_2}

鼻クリップしたシンクロ選手。
美人であればあるほど。
\subsubsection{\textbf{DONE}}
\label{sec-81_1_3}

瞬間接着剤が発売された当時。

「ほんとに瞬間でくっつくのかよ」と自分の指先で実験。

結果えらいことになった。

あの頃みんな通った道かも。
\subsubsection{\textbf{DONE}}
\label{sec-81_1_4}

女の子の前でカッコつけてグラスを口へ運ぶとき、
コースターがくっついてくる。
\subsubsection{\textbf{DONE}}
\label{sec-81_1_5}

レンタルビデオ店でエロビデオを借りようとして、

「これ以前も借りてますよ」

「いいんです、また借ります」
と云うとき。
\subsubsection{\textbf{DONE}}
\label{sec-81_1_6}

わが子にふざけてキスマークを付けられた。
\subsubsection{\textbf{DONE}}
\label{sec-81_1_7}

階段の上の方にいる女子高生のスカートに気を取られ、
階段の下の方ですっこけるオッサン (俺)
\subsubsection{\textbf{TODO}}
\label{sec-81_1_8}

「芸能人いらない情報コンテスト」

「樽ドル」でネット検索かけると楽しいです。
\subsubsection{\textbf{TODO}}
\label{sec-81_1_9}

ロケットマンとはふかわりょうの別の顔だ!!!!!

ついさいきん知りました。
\section{2011-08-18}
\label{sec-82}
\subsection{子供みたいなことで喜ぶ大人の会 // 全裸の世界}
\label{sec-82_1}
\subsubsection{\textbf{DONE}}
\label{sec-82_1_1}

子供みたいなことで喜ぶ大人の会

小田急線沿線に住んでいるので、ロマンスカーが通るとテンションが軽く上がります。
チビたちと「ロマンスカーだっ!」と叫びます。
\subsubsection{\textbf{DONE}}
\label{sec-82_1_2}

チビたちとホットケーキを作ってたとき。

玉子をボールに割ったら双子であったのでみんなで大騒ぎに。
そのとき出掛けていたヨメサンに「やったよ!」ってメールしました。
\subsubsection{\textbf{DONE}}
\label{sec-82_1_3}

日焼けにより皮膚から剥れた皮がすごく大きかった。
\subsubsection{\textbf{TODO}}
\label{sec-82_1_4}
\section{2011-08-17}
\label{sec-83}
\subsection{エ小学生 // 選曲家養成ゼミナール}
\label{sec-83_1}
\subsubsection{\textbf{DONE}}
\label{sec-83_1_1}

「エロ小学生」

小さな商店の店先の小さな本のコーナーは昔からありました。
そこで雑誌巻頭のヌードグラビアを片っ端から開いてまじまじと見てました。

隣りで立ち読みしてたオッサンは (エロ小学生がっ) と思っていたに違いありません。

さいわいお店の人に止められたことはありませんでした。
\subsubsection{\textbf{DONE}}
\label{sec-83_1_2}

小学生の頃、親戚の家に寄った際のこと。

子どもが居ない家だったので遊ぶものがなくあちこち物色していると、ふとテレビの上の本に目が止まりました。
おじさんの本でした、表紙には見たことない状態の女性の写真。
縄がかかってて吊るされていました。
内容もオールカラーでそんな感じの女性の写真ばかり。

今思うと、そういった本をそのへん出しっぱなしで平気な人だったんですね、おじさんは。

「そんな本見ててしょうがないねぇ、ごめんねぇ」おばさん(おじさんの母上、僕の大叔母さん)はそう云いますが別にその本を片付けようとするわけでもなく。

なので母も「そんなもの見るんじゃありませんっ!」と取り上げるわけにもいかなかったようです。

僕は全ページじっくりとと拝見いたしました。

もちろん現在までの性的指向に多大な影響があったのは云うまでもありません。
\subsubsection{\textbf{DONE}}
\label{sec-83_1_3}

父のエロ本を発見したとき、どうしても欲しくなり。

1ページくらい無くなっても気付かないよね、と自分に云いきかせ、一番最初のページを (ペリペリ) と破いてポケットへ入れました。

何度考えても、なぜ気付かれないと思ったのか分かりません。
\subsubsection{\textbf{TODO}}
\label{sec-83_1_4}

胸の大きい子がいると胸ばっかり見てましたね。
私はその頃勉強も運動もできた方だったので、それなりに優等生だったと思いますが。
今考えると女の子の評価は低かったんではないかと類推します。
\section{2011-08-10}
\label{sec-84}
\subsection{ずるい自分大公開 // 選曲家養成ゼミナール}
\label{sec-84_1}
\subsubsection{\textbf{DONE}}
\label{sec-84_1_1}

さいきん、自分の送るネタは所帯染みてるため GLZ向きではないのではと分かり始めた来週で40歳 りとるぐれいです。

「ずるい自分大公開」ですが、

毎月定期券を購入しています、購入時期に土日またぎや連休にかかっていたりすると、買うのを数日延ばします。
それが積もり積もって給料日を越えると、まるまる定期代がおこづかいになります。

ヨメサンはこのシステムのことは知りません。
と云っても半年に一回あるかないかの話なので、純粋に自分のボーナスみたいな感じです。
(ちなみに本物のボーナスはほとんど手元にゃ来ませんので (泣)

やっぱり所帯染みたネタでした、すみませんな。
\subsubsection{\textbf{DONE}}
\label{sec-84_1_2}
\section{2011-08-09}
\label{sec-85}
\subsection{中山連想地獄 // 金持ち伝説}
\label{sec-85_1}
\subsubsection{\textbf{TODO}}
\label{sec-85_1_1}

中山連想地獄
\subsubsection{\textbf{DONE}}
\label{sec-85_1_2}

灼熱 575

タンクトップ 谷間にたまる 汗しずく
\subsubsection{\textbf{DONE}}
\label{sec-85_1_3}

信号待ち うなじに貼りつく 髪ひとすじ

宅配便 出てきた人妻 足あらわ
\subsubsection{\textbf{TODO}}
\label{sec-85_1_4}

コンビニまで 履かずに出かける 団地妻
\section{2011-08-08}
\label{sec-86}
\subsection{欠点図鑑 // 狭い所クイズ}
\label{sec-86_1}
\subsubsection{\textbf{DONE}}
\label{sec-86_1_1}

欠点図鑑

カエルがだめです。
こうやって文字に書くのもいやです。

子ども番組を子どもと見ていてアレが映ると「ヒィィィィ」て云います。
ちなみに来週40歳になります。

Twitterで、自分のアイコンをアレにしている昔の友人がいるのですが、ホントはアンフォローしたいです。
ていうかスキをみてします。
\subsubsection{\textbf{DONE}}
\label{sec-86_1_2}

ドラえもんの欠点。

動力は核融合炉。
***
\subsubsection{\textbf{TODO}}
\label{sec-86_1_3}

カレーは大好きだが、皿と鍋洗いがねー
\section{2011-08-02}
\label{sec-87}
\subsection{田舎ランキング}
\label{sec-87_1}
\section{2011-08-01}
\label{sec-88}
\subsection{スポーツにつっこもう!// 芸能人勝手にアンケート}
\label{sec-88_1}
\subsubsection{\textbf{TODO} スポーツにつっこもう!}
\label{sec-88_1_1}

「スポーツにつっこもう!」
\section{2011-07-27}
\label{sec-89}
\subsection{所詮、動物 // 夢の値段}
\label{sec-89_1}
\subsubsection{\textbf{DONE}}
\label{sec-89_1_1}

「所詮、動物」

動物的というのともちょっと違うのですが。
ウチの2歳のチビが、部屋の中あらぬ方向を見ながら手を振っているのを見ると、
なにか野生的なもの感じます。

怖いのです。
\section{2011-07-26}
\label{sec-90}
\subsection{美男美女は経験しない事 // 検索ワードクイズ}
\label{sec-90_1}
\subsubsection{\textbf{TODO}}
\label{sec-90_1_1}
\section{2011-07-25}
\label{sec-91}
\subsection{テレビにつっこもう! // 東京ミニマム情報}
\label{sec-91_1}
\subsubsection{\textbf{TODO}}
\label{sec-91_1_1}

「テレビにつっこもう!」

昨日ついにTVがデジタル放送に切り変わったわけですが。
家のTVはアナログのままです。

きのうの正午前はTVを点け某スペシャル番組にて切り替え後はこんな画面になりますという表示を確認。

カウントダウンが始まるとヨメと二人、ブルーバックの画面になるのをちょっとワクワクしながら待っていました。

AM0時。
なりませんブルーの画面に。

がっかりしました。
どうやらマンションで入っているケーブルTVがアナログTVの人たち用に、一括してチューナーに通してくれているらしいのです。
右上の「デジアナ変換」というのが目障りな以外はこれまで通り見られてます。

ヨメサンと「なんだツマラン!!」とTVに向かって叫んだのでした。
\section{2011-07-21}
\label{sec-92}
\subsection{ピスと過ごした夏休み}
\label{sec-92_1}

りとるぐれ子 24歳 です。

「ピスと過ごした夏休み」

ピスちゃん、ご・ぶ・さ・た、ぐれ子です。
ピスちゃんと過ごした夏の思い出、話しちゃっていいのね。

あの夏のことアナタは覚えているかしら?

二人で浴衣を着て行った、花火大会の夜。
河原を腕を組んで歩いたわね。
そして人気の無い土手を見つけて草の上へ並んで座ると、
花火に目をやりながら肩にそっと手を回してきたピスちゃん。
それから私のクチビルを、、、

きゃーきゃーはずかしい!
今年も会えるかしら。。。
\section{2011-07-20}
\label{sec-93}
\subsection{心が死んだ日 / 大人のための夏休み小学生講座}
\label{sec-93_1}
\subsubsection{\textbf{DONE}}
\label{sec-93_1_1}

葉山レイコ
\subsubsection{\textbf{DONE} ヽ(゚∀゚)ノ 10回目 記念か。。}
\label{sec-93_1_2}

「大人のための夏休み小学生講座」

りとるぐれい 39歳
アンコが食べられません。
どうしたら食べられるようになりますか
\section{2011-07-14}
\label{sec-94}
\subsection{一度した事はニ度する / ゆる雑テキトー ランキング}
\label{sec-94_1}
\subsubsection{\textbf{DONE}}
\label{sec-94_1_1}

「一度した事はニ度する」

僕の風呂で体を洗う順番は、

 体 → 顔 → 髪

なのですが。
たまにきまぐれて髪を先に洗ったりすると、

 髪 → 体 → 顔 → 髪

とループします。けっこうします。。。 来月40なので許してください。
\subsubsection{\textbf{DONE}}
\label{sec-94_1_2}

ヨメサンと話していて、

あれ、この話って前もしたよね。それで、同じ結論に逹したよね。
ということがよくあります。

二度くらいなら良いのですが、議案によっては4、5回平気でしてます。

自分は来月40なので許してもらうということで、仕方ないのですが。
ひとまわり下のヨメサンの脳が心配です。
\subsubsection{\textbf{DONE}}
\label{sec-94_1_3}

すでに買っている漫画を、また買ってしまうということは、30過ぎたあたりから
ちょこちょこあるじゃないですか。

さいきんは2冊目を買って、読み終ってもまだ気付かないことがあります。
2度楽しめておトクですね。

来月40なので許してください。
\subsubsection{\textbf{DONE}}
\label{sec-94_1_4}

さてプールの季節ですが。

昨年のことチビたちと「Υみうりらんど」へ行ったときのこと。
しばらく遊んでいたら、自分の水着のお尻の部分が少し破れていることに気付きました。

それでおととし、同じようなシチュエーションで
「さきおととしにウォータースライダーで破った尻」を思い出したことを思い出しました。

たぶん今年もプールまで行ってから気がつくと思います。
来月40なので許してください。
\section{2011-07-13}
\label{sec-95}
\subsection{暑さにやられた私 / 少し殺す}
\label{sec-95_1}
\subsubsection{\textbf{DONE}}
\label{sec-95_1_1}

「暑さにやられた私」

残業の夜。

一人なのに広いオフィス内、クーラーを効かせるのも気が引けてOFFに。

窓を開ける。

でもそりゃ暑い。

PCの壁紙を水着の姉ちゃんにした。

ちょっと涼しいような。。。 気になった
\subsubsection{\textbf{DONE}}
\label{sec-95_1_2}

朝食後トイレに入ったら汗がしたたり落ちてきた。

ヨメサンが正面に貼った「みつをカレンダー」の内容に励まされた。

『つまずいたって
    いいじゃないか
      にんげんだもの』

普段は見向きもしないのに、なんか来るわ。
\subsubsection{\textbf{DONE}}
\label{sec-95_1_3}

昨夜、Macのファンが回りっぱなしでした。

この娘のイイ声が聞こえないじゃないか。
\subsubsection{\textbf{DONE}}
\label{sec-95_1_4}

淫夢を見ました。
\subsubsection{\textbf{DONE}}
\label{sec-95_1_5}

今年は日射しが異常に感じます。

先週は熱中症で2日仕事を休みました。
3日目も体調はけっして良くなかったのですが、重要なミーティングがあったため
体を引きずって駅に向かったのですが。

階段を登っていると目の前にホットパンツの白い足。
フラフラとその足についていき反対の車線の電車に乗りこみました。
なんとかドアが閉まる前に気付いて降りました。

体調が悪いと地が出るんですな。
\subsubsection{\textbf{DONE}}
\label{sec-95_1_6}

「暑さにやられた私」

汗っかきなのでタオルを次々と出しては汗を拭っているのですが。

手元からすぐに無くなって、またすぐに出してきて拭っていると、

そこらじゅうにジブンのタオルだらけになります。
\subsection{少し殺す}
\label{sec-95_2}
\subsubsection{\textbf{DONE}}
\label{sec-95_2_1}

「少し殺す」

・あんな顔なのにどうしていつもミニスカートなんだ

・いつも視線が生え際あたりをさまよっているんだよ

・「耳毛生えてんな」と云われる

・コンビニにて「T○○カードはお持ちですか?」それ今日3回目、無いよ
\subsubsection{\textbf{DONE} ヽ(゚∀゚)ノ ひさびさ 9回目}
\label{sec-95_2_2}


こちらが食べているものみて「ああ、それ、美味しくないよね」て云うやつ
\section{2011-07-12}
\label{sec-96}
\subsection{輪廻2011 / バーベキューバカ大興奮!}
\label{sec-96_1}
\subsubsection{\textbf{DONE}}
\label{sec-96_1_1}

「輪廻2011」

5歳になるチビが朝食にて、ぐるぐると混ぜていた納豆から箸を離し。

その先をじっと見ると、

「洗って」
とこちらに箸を突き出した。

変なところばっかり自分の子どもの頃にソックリだよ。
\section{2011-07-11}
\label{sec-97}
\subsection{所詮な話 / 反射神経選手権}
\label{sec-97_1}
\subsubsection{\textbf{DONE}}
\label{sec-97_1_1}

「所詮な話」

子どもの頃飼っていた犬の「チロ」ですが。
たいへん可愛がっていたし、もちろんチロも僕にすごくなついていました。

ある日父がカビが生え始めたパンをチロに投げ与えました。
父の感覚としては、動物なんだから多少のカビくらいは平気なはずだと、捨てるよりは良いだろうとあげたようなのですが。

それを見た僕はというと「なにそんな物あげてんだよ! なんかあったらどうすんだよっ!!」
と慌てて彼の咥えていたパンを取り上げようとしました。

しかし、、、、
結果としてチロは僕に牙をむいたのでした。ガブリと手をやられました、流血しました。
僕は硬直。。。

「所詮はケモノか」と今でもちょっと悲しくなる思い出です。
\subsubsection{\textbf{DONE}}
\label{sec-97_1_2}

さいきん実家の近くに住む、僕の姉が一人でこちらに来て一泊していったのですが。
ヨメサンと姉と二人で飲んでいたときに漏らしたらしいのですが、その内容が僕には衝撃的でした。

義兄は温厚な人で、両親とも仲が良く、甥っ子たちにも慕われているという、良いイメージばかりだったのです。まあ多少は姉から文句が出ているのは聞いていたのですが。

その時の姉いわく、
「あんな向上心の無い人とは老後一緒に住む気がしない」
「まったく老後一緒にいる場面をイメージできない」
「あたしは子どもさえいればいいの」
「あたしは一人で老後を過ごすの」

と淡々と語っていたとか。
勿論そんな思いは義兄はまったく知らないらしいです。

「夫婦といえど所詮は他人か」といささかショックを受けたのでした。

そして自分はヨメサンに愛想つかされていないかと、彼女に不自然に優しくなったようです。
\subsubsection{\textbf{DONE}}
\label{sec-97_1_3}

今日の昼休み、引き落し・振込み手数料を取られないために、銀行、ATM、郵便局をこの炎天下ハシゴしました。

所詮、庶民だし、、、 と思いました。

「手数料とか細かいこと気にしないぜ!」っていう性格と経済力に憧れます。
\subsubsection{\textbf{DONE}}
\label{sec-97_1_4}

仮面ライダーなんて、所詮バッタじゃん。

と思ったら今度始まる、新ライダーはイカっぽいです。
\subsubsection{\textbf{DONE}}
\label{sec-97_1_5}

CDシングルの両A面って所詮二曲しか入ってないよという意味しかないと思うんだ。
\subsubsection{\textbf{DONE}}
\label{sec-97_1_6}

JK(女子高生)とか呼ばれるのは所詮3年間だけだぞ、旬が過ぎないウチに男つかまえとけよっ
\subsubsection{\textbf{DONE}}
\label{sec-97_1_7}

マルモなんて所詮イヌだろ、と思っていたら、犬の名前じゃなかった。所詮オヤジなんて。。。
\section{2011-07-05}
\label{sec-98}
\subsection{夏のホラー特集 (ホラだ)}
\label{sec-98_1}
\subsubsection{\textbf{TODO}}
\label{sec-98_1_1}

夏のホラー特集

クニマスの発見で有名になった東京海洋大学客員准教授さかなくん。

魚好きのため魚を食べないのでは、との質問をよくされるが、魚を食べるのは大好き。

ただ自分に顔が似ていると思いこんでいる「まんぼう」だけは食べられない。
\section{2011-07-04}
\label{sec-99}
\subsection{黒いあるある/中山ガッカリコンテスト}
\label{sec-99_1}
\subsubsection{\textbf{DONE}}
\label{sec-99_1_1}

「黒いあるある」

電車でこれみよがしに化粧している女性は、だいだい B U S U
\subsection{中山ガッカリコンテスト}
\label{sec-99_2}
\subsubsection{\textbf{TODO}}
\label{sec-99_2_1}
\section{2011-06-30}
\label{sec-100}
\subsection{サマーサクセス/アーティストの名言}
\label{sec-100_1}
\subsubsection{\textbf{TODO}}
\label{sec-100_1_1}
\section{2011-06-29}
\label{sec-101}
\subsection{GLZ 本音祭り/オッサン矯正クイズ}
\label{sec-101_1}
\subsubsection{\textbf{DONE}}
\label{sec-101_1_1}

「本音祭り」

キャバ嬢「私たちもお酒頼んでいいですか〜」
僕「もちろん! なんでも頼んでよ!」

(おいおい、高いのはやめてくれよ〜)
\subsubsection{\textbf{DONE}}
\label{sec-101_1_2}

「まぁ僕もどっちかっていうとふくよかな方が好きだなぁ」

(巨乳じゃねぇ、ただのデブじゃねぇか)
\subsubsection{\textbf{DONE}}
\label{sec-101_1_3}

床屋にて。

女性店員洗髪しながら
「どこかかゆいところありますか?」

僕「い、いえ、、大丈夫です」
本音(股間がカユいんだよ、ムネがさっきから頭にあたるんだよー)
\subsubsection{\textbf{DONE}}
\label{sec-101_1_4}

満員電車にて。

「ああ! すみません大丈夫でしたか?!」
足を踏まれた僕「あいえ、大丈夫です。平気です」

本音(どうせなら他の所を踏んでくれ)
\subsubsection{\textbf{DONE}}
\label{sec-101_1_5}

8月は、ヨメチビたちが僕の実家、ヨメサンの実家の両方へ長期間出かけるのですが。

僕はというともちろん仕事がありますので「8月は淋しいよ」とか云っていじけたフリをしているのですが。

実際のところ
「フリーダム!!!」と思ってます。今から楽しみです。
\section{2011-06-28}
\label{sec-102}
\subsection{GLZ 本格派に見える方法/ガッカリアンケート}
\label{sec-102_1}
\subsubsection{\textbf{TODO}}
\label{sec-102_1_1}

「本格派に見える方法」

牛丼屋に行き「牛丼ねぎだく」を頼む
\subsubsection{\textbf{TODO}}
\label{sec-102_1_2}

「本格派に見える方法」
iPhone
キーボード
通信してるんだよ

ホワイトソースではなく、ベシャメルソースと云う
\section{2011-06-27}
\label{sec-103}
\subsection{GLZ これで金取るの?}
\label{sec-103_1}
\subsubsection{\textbf{DONE}}
\label{sec-103_1_1}

「これで金取るの?」

小学生低学年の夏休み、親戚の家に家族で旅行に行きました。

そこで地元のお祭に参加。
人生初めての「お化け屋敷」に入ることになりました。
狭い通路を姉と従兄弟と歩いて行ったのですが、幼なかった自分は二人の手をしっかり握っていたと思います。

しかし入口のところで作りものの大きな顔のお化けにちょっとビックリした以外は、暗い通路の中特になにも起こらず。

出口の近くに来ると、お化けの扮装をしたオッサン二人が「お客に足踏まれちゃったよー」と大口開けて笑ってました。

これで金取るの? と思いながら外に出ました。
待っていた母に「面白かった?」と訊かれましたが、子どもながら苦い顔をした覚えがあります。
自分のお金で入ったわけでもないのに「金返せ!」と思いました。
\subsubsection{\textbf{DONE}}
\label{sec-103_1_2}

専門学生時代の話です。

いつもツルんでいるグループの一人、Hのの家は小さな焼き肉屋をやっていました。
ある日どういう流れからか分かりませんが、そのHの家で焼き肉をやろうということになりいつものメンバーで彼の家に行きました。

夕食の時間なのに僕ら以外にお客は無く。
僕らは貸切り状態に気を良くしてどんどん食べもの、飲みものを注文しました。
また、Hのばあちゃんが「これ珍しい肉だから食べな」と出してくれるものを、出されるままに口に入れていました。
正直、そんなに口が肥えていない僕らでも「ふつーだね」と小声でつつき合うクオリティだったのですが。

そんな調子だったのでHを含め8人いたメンバーの胃はあっというまにいっぱいになり、気持ちの良い呈で家に帰りました。
かなり盛り上がったのでまたやろうぜ、という話をして別れたのですが。

数日後、Hから請求が決ました。一人 8千円。
メンバー半分は女性だったのですが同じ額の請求が来たとのこと。

確かに「全部奢ってくれないにしても負けてはくれるだろう」と、全員甘いことを考えていたと思いますが、学生ですし、みんなバイトでこずかいを稼いでいた頃でしたので8千円はけっこうな額です。

そんなに高いの? とも 負けてよ とも云えず、なんとなく納得行かない顔でみんなHにお金を払ったのですが。

これで済んでいれば、おそらく高い肉を沢山食べたんだなぁ、と自分に云い聞かせる感じで終ったと思うのですが。

しかし後日、当時同グループ内にはHがつき合っていた彼女がいたのですが、彼女からは一銭ももらっていないことが発覚。

それでリーダー格だったSが、Hにつめ寄りました。
「なんだよお前の彼女の分までこっちがかぶって払ったのかよ」
「もしかしてお前の分まで払ったのか」などとなり
しまいには
「そもそも俺たち、そんなに食ってねぇ!」と。

それでも払ったものが返ってくるわけでもなく。

しばらくHとは話しませんでしたね、一年くらい。。。
\subsection{お金575}
\label{sec-103_2}
\subsubsection{\textbf{DONE}}
\label{sec-103_2_1}

「お金 575」
ボーナスが 出たのに手元にゃ 残らない

おこづかい 底をつくよな むだづかい
\subsubsection{\textbf{DONE} ヽ(゚∀゚)ノ 8回目}
\label{sec-103_2_2}

へそくりを 子どもが見つけ ヨメのもの
\subsubsection{\textbf{DONE}}
\label{sec-103_2_3}

10年前の キャバクラ通い 試算する
\subsubsection{\textbf{DONE}}
\label{sec-103_2_4}

秋元が おたくの金を 吸いとるよ!
\subsubsection{\textbf{DONE}}
\label{sec-103_2_5}

ハンカチで 出逢いを演出 団地妻
\subsubsection{\textbf{DONE}}
\label{sec-103_2_6}

Fカップ 谷間に札の 夢をみた
\section{2011-06-23}
\label{sec-104}
\subsection{GLZ 業界地獄絵図}
\label{sec-104_1}
\subsubsection{\textbf{DONE}}
\label{sec-104_1_1}

「デスマーチ」という言葉があるとおり、IT業界はときに地獄の縁を覗き込むことがあります。
僕も何度かあります。あやうく戻って来れているのですが。

いつしか消えていく人。遁走話もよく聞きます。
\subsubsection{\textbf{DONE}}
\label{sec-104_1_2}

「業界地獄絵図」

以前、某大手警備会社に勤めておりました。
僕はさいわい最初から情報系担当スタッフとして入社したので良かったのですが。

僕たちより前に入社した先輩たちはまず警備の仕事をひととおりすることになっていました。

それで常駐警備の仕事をしていた先輩の話ですが。

銀行の深夜警備中のこと。
ある日、手違いから銀行入口の二重の自動ドアの間に閉じこめられてしまったそうです。

一人では中からどうにもできないらしく、腹を決めて朝を待つことにしたらしいのですが。
連日のシフトの激務から、いつしか床で横になって眠りこんでしまったらしいです。

朝シャッターが上がり、自動ドアも作動し、最初のお客様に発見されたそうです。

もちろん、すごい怒られたそうです。
\subsection{全日本SEリクエスト (こんな音ありますか)}
\label{sec-104_2}
\subsubsection{\textbf{DONE}}
\label{sec-104_2_1}

「全日本SEリクエスト」

退陣を決めた菅さんの「歯はぎしり」ありますか?
\subsubsection{\textbf{DONE}}
\label{sec-104_2_2}

近藤夏子の腰骨の音
\subsubsection{\textbf{DONE}}
\label{sec-104_2_3}

中山明日実の肩幅に耐えきれずTシャツが破れる音
\subsubsection{\textbf{DONE}}
\label{sec-104_2_4}

ストッキングの破れる音
\subsubsection{\textbf{DONE}}
\label{sec-104_2_5}

ラブホテルやり逃げ高校生カップルの2階から飛び降りたときの着地音
\subsubsection{\textbf{DONE}}
\label{sec-104_2_6}

中山嬢がピストンさんの坊主アタマを叩く音。
\subsubsection{\textbf{DONE}}
\label{sec-104_2_7}

電車内、アフロの頭から出ている湯気の音。
\subsubsection{\textbf{DONE}}
\label{sec-104_2_8}

僕のメガネがヨメの靴の下でつぶれた音。
\subsubsection{\textbf{DONE}}
\label{sec-104_2_9}

灰皿テキーラからの海老蔵音。
\subsubsection{\textbf{DONE}}
\label{sec-104_2_10}

木村カエラの出産音。
\subsubsection{\textbf{DONE}}
\label{sec-104_2_11}

ためぐちローラのベロ音、てありますか?
\subsubsection{\textbf{TODO}}
\label{sec-104_2_12}

草食男子が肉食女子に食われる音。
たぶん ナンバー 1 です。
\section{2011-06-22}
\label{sec-105}
\subsection{GLZ ドン引きマスター}
\label{sec-105_1}
\subsubsection{\textbf{DONE}}
\label{sec-105_1_1}

先週末の休みのことでございます。

長男、次男がTVに夢中になっていたので、ヨメサンと寝室にしけこみました。
まぁ、部屋でなにをしてたのかは明記しませんが、仲良くやってました。

チビたちの行動パターンは把握してますので、もしこちらに近付いて来る場合には、話し声とか廊下の足音とかで察知できます。
僕は時々ドアの方を気にしていたのですが、ヨメサンは「大丈夫、鍵かけてあるから」とか云ってます。

それでなんやかやしてたんですが、なにか外の様子がおかしい、部屋の周りでうろちょろしてる気配がするのです。

ノックがされて「なにやってんのー」と声がかるのを待ってみたのですが、それもなく、
また静かになったので、とめていたことを再開。

と思ったら、鍵をしたはずのドアがゆっくりと開きました。
そして、5歳児と2歳児がそこに。。。
「なにやっての?」と長男。
両親絶句。

さいわい変なところは見られなかったのですが。

5歳の長男は寝室に鍵がかかっているのを見ると、隣のクローゼットに入り自分の財布から50円玉を取り出して外から鍵を回したのでした。一切声を出さす。
次男はニイチャンなんか面白いことやってんなと見てたのでした\ldots{}

いつかはこんなことやめなきゃいけないな、と思っていたのですが、早かったなぁー

わが子の行動にどん引きしました。というかホラーです。
\subsubsection{\textbf{TODO}}
\label{sec-105_1_2}
\section{2011-06-21}
\label{sec-106}
\subsection{GLZ 今じゃ考えられない事}
\label{sec-106_1}
\subsubsection{\textbf{DONE}}
\label{sec-106_1_1}

「今じゃ考えられない事」

ラジコが無いこと。

かつては、住む環境によって電波がうまく入るラジオ局を聴くしかありませんでした。

ラジコが無ければ GLZ を聴くことも無く、「ゲリラ来てるの!!」と会社をそっと抜け
出して、駅に走ることも無かったと思います。
\subsubsection{\textbf{DONE}}
\label{sec-106_1_2}

「今じゃ考えられない事」

ウチにはTVの録画装置がありません。
数年前に僕がHDDレコーダーのフタを開けて内部を掃除をしていたら壊しました。。。

そして、しばらく録画しない生活をしていたら「まあいいや」てなってしまいました。

TVはリアルタイムで見る! のみ。
別に支障はありませんでした、自然にTV見なくなりましたし。

それで、まだTVはアナログです。
ヨメサンと「実際映らなくなるところを見てやろう」、とか云ってます。
チビたちは泣きます、きっと。。。

こんな家庭、いまどき考えられませんかね。
\subsection{世界のコレクター}
\label{sec-106_2}
\subsubsection{\textbf{DONE}}
\label{sec-106_2_1}

世界のコレクター

爆笑太田さんの奥様、タイタン社長の光代さんは。
子供の頃からの、切った爪や、剥いた皮膚の皮、かさぶた等をビンに、ずっとためてい
るらしいです。
なんかヤダ。
\subsubsection{\textbf{DONE} ヽ(゚∀゚)ノ 7回目}
\label{sec-106_2_2}

世界のコレクター

ご近所さんでクマのぬいぐるみコレクターさんがいます。
てっきり奥様が好きなんだと思ったら、集めているのは旦那さんらしいです。

部屋一面にクマさんです。
聞けば、毎日配置を変えているんだとか。

お宅に行ったときドアが開いていて見えると、けっこう壮観です。
夜はちょと怖いのですが。
\subsubsection{\textbf{TODO}}
\label{sec-106_2_3}
\subsection{KIRA}
\label{sec-106_3}

小島さん、神足さん、こんにちは

TV・エアコン・冷蔵庫 家電の世界へようこそ

私はシステムエンジニアをしており、機械にはそこそこ強い方だと自負しておりました。
自作のパソコンを組み立てたり、調子の悪くなったTVを開いて接触不良を直すとか、ビデオデッキをバラして出てこなくなってしまったお気に入りAVテープを救出するとかは普通にしておりました。

そんな感じでしたが、ある時からもうそういったことはやめようと思うことがありました。

なぜかというと、「集積度がどんどん上がってきた」ということです。
小型化につぐ小型化で、機器はどんどん小さくなっていくのですが、部品性能が上がっ
たということはもちろんあるのでしょうが、
昔のようにゴキブリが家電を巣にしている、ということもすでに無くなっていることで
しょう。
\section{2011-06-17}
\label{sec-107}
\subsection{KIRAKIRA いろんな奥さんの話}
\label{sec-107_1}
\subsubsection{\textbf{DONE}}
\label{sec-107_1_1}

うちの奥さんの話なんですが。

お子さんが生まれたとたんに、パパママになってしまう夫婦って多いみたいです。

自分は子どもたちの前以外ではヨメサンの事を名前で呼ぶようにしています。
もちろん本人にも。
これってけっこう大事なことと思っています。

オバさんにならないで、女を捨てないで、という思いがこもってますかね。
もちろん僕もオッサンにならないように努力してますよー。

小島さん、博士はお互いのことどう呼んでいますか?
\subsubsection{\textbf{DONE}}
\label{sec-107_1_2}

友だちの奥さんの話です。

その友だち夫妻は「バツいち」同士で、お互いすったもんだの挙句に結婚したという経緯がありまして。

不倫中、旦那の会社に当時の奥さんが怒鳴りこんできたとか\ldots{}
家に帰ったらフォークが机に刺さっていたとか\ldots{}

それでもまぁなんとか一緒になれて、現在は子ども生まれて幸せにやっているようなのですが。

結婚してからしばらく経ったある日、旦那の前の奥さんと一緒にお揃いで買ったという携帯電話が出てきたらしい。
そこで、

 奥様「なによコレ!」
 旦那「\ldots{}あ、もう要らないよ」
 奥様「捨てていいの?」
 旦那「\ldots{}うん、捨てて」

奥様は怒りが再燃したらしく、その携帯をカレーの鍋に入れて煮込んだらしいです。

なぜそんなことをしたのか凡人の身にはよく分かりませんが、女の怒りは恐しいと感じました。
\subsubsection{}
\section{2011-06-15}
\label{sec-108}
\subsection{GLZ 男と女のセコさ合戦}
\label{sec-108_1}
\subsubsection{\textbf{DONE}}
\label{sec-108_1_1}

ピストン兄さんに年をバラされた39歳男です。

「男と女のセコさ合戦」

昨年、新入社員の男の子から聞いた話です。
彼はついさいきんまで、個室の漫画喫茶でバイトをしていたとのこと。
よく金の無い学生カップルが来店して、個室をいわゆるそういう場にしてしまっていることがあったそうです。
部屋から飲み物などの注文があったとき、タイミングを見計らって入って行くとかなり高確率でそういう場面を見られたそうです。
お金が無くてホテルを借りられないんでしょうね。まあ若い二人はどこでも良いってことでしょうけど。

それを聞いたとき特に興奮することもなかったのですが、すでに妻子もいたし、年が年ですし。。。
ただその夜は夢に見ました。
\subsubsection{\textbf{DONE}}
\label{sec-108_1_2}

男子学生が二人、ピッタリくっついて改札を抜けていくのを見たことあります。
何人まで行けるんだか。。。
\subsubsection{\textbf{DONE}}
\label{sec-108_1_3}

ラブホで彼女がお腹が空いていたらしく次から次へと食べ物を注文。
一度に頼めばいいのに何度も何度も、今もあるのか知りませんがエアーのパイプでメモを送っていました。

間のびしたこともあって、こちらはなんか冷めてしまって、コトがうまいこと運ばず。

それで、コンドームの催促をしたら、パイプを経由して「シュパッ」と届きました。
\section{2011-06-14}
\label{sec-109}
\subsection{GLZ 宇宙の真理}
\label{sec-109_1}
\subsubsection{\textbf{DONE}}
\label{sec-109_1_1}

おっぱいより尻に
\subsubsection{\textbf{DONE}}
\label{sec-109_1_2}

お気に入りの傘ほどわすれる
\subsubsection{\textbf{DONE}}
\label{sec-109_1_3}

薄毛。気にするのをやめたら進行がピタっと止まった
\subsubsection{\textbf{DONE}}
\label{sec-109_1_4}

「そんなことあるわけねーじゃん」と思ってたが体育のマットの耳でケガしたやつがいた
\subsubsection{\textbf{DONE}}
\label{sec-109_1_5}

テキストベースのブラウザで見ると

NAVIGATOR ピストン西沢 秀島史香

のままです。
\subsubsection{\textbf{DONE}}
\label{sec-109_1_6}

重要なパスワードほど。。
\subsubsection{\textbf{DONE}}
\label{sec-109_1_7}

ダイエットコークを飲んでる人は太っている
\subsubsection{\textbf{DONE}}
\label{sec-109_1_8}

「Tシャツ欲しい、Tシャツ欲しい」と思いつつネタを書いていると、幾つ出しても読まれないんだ、きっと
\subsubsection{\textbf{DONE}}
\label{sec-109_1_9}

ぶすっ子が書く自画像は、4割り増し。
\subsection{後ろめたいアンケート}
\label{sec-109_2}
\subsubsection{\textbf{DONE}}
\label{sec-109_2_1}

後ろめたいアンケート

パンツは三日まで行ける
\subsubsection{\textbf{DONE}}
\label{sec-109_2_2}

仕事しながら GLZ にメールを連投している
\subsubsection{\textbf{DONE}}
\label{sec-109_2_3}

風呂に3日は入らない

基本背中は洗わない

スカートの女性が階段を登っていたら、つい後ろに回る (見えはしないけど

仕事中、エロサイトを見ていてネットワーク管理者から注意されたことがある
\subsection{GLZ人探し}
\label{sec-109_3}
\subsubsection{\textbf{DONE}}
\label{sec-109_3_1}

人探し

「キミの瞳に乾杯」と真顔で女性に云ったことのあるひと。
\subsubsection{\textbf{DONE}}
\label{sec-109_3_2}

実は、中山明日実がすごいタイプである。
\subsubsection{\textbf{DONE}}
\label{sec-109_3_3}

女装して街を歩いたことがある。
\subsubsection{\textbf{DONE}}
\label{sec-109_3_4}

父親のかつらをつけてみたことがある人いますか?
\subsubsection{\textbf{DONE}}
\label{sec-109_3_5}

ネットで話題になった、コーモン様にミンティアをたくさん入れたことがある人。
\subsubsection{\textbf{DONE} ヽ(゚∀゚)ノ 6回目 サラリとだが}
\label{sec-109_3_6}

スライダーで破れたハナシ
\subsubsection{\textbf{TODO}}
\label{sec-109_3_7}
\section{2011-06-10}
\label{sec-110}
\subsection{CIRCUS2 下北沢}
\label{sec-110_1}
\subsubsection{\textbf{TODO}}
\label{sec-110_1_1}

30歳になったあたり、笹塚に住んでおりましたのでよく下北には歩きで行っておりました。

今のヨメサンとつき合い始めた頃でしたので、二人で飲み歩いてましたね。

それで飲み屋さんではないのですが、
当時ヨメサンと二人、お気に入りでよく行っていたお店が、

 THE STUDY ROOM スタディ・ルーム (\href{http://www.thestudyroom.co.jp/}{http://www.thestudyroom.co.jp/})
 
でした。

子ども向けの学習キットなどを売っているお店なのですが、大人二人でいろんな物をおもちゃ感覚で触り倒していました。
店員さんがたいへん優しく対応してくれるので、ついつい余計なものまで買ってしまったり。

つい先週末、初めて子どもたちを連れて行ってきました。
チビたちも気に入ってくれたみたいです。キャッキャ云って遊んでましたねー。

「また行く!」ということになりそうです。
\subsection{KIRA 当選・落選 いろんな選挙の}
\label{sec-110_2}
\section{2011-06-09}
\label{sec-111}
\subsection{GLZ 世の中の不思議}
\label{sec-111_1}
\subsubsection{}

谷亮子議員を誰が最初にヤワラちゃんと呼んだの?
***
中山あすみネタはリスナーがわかり易く結束する
\subsection{男のゼクシィ}
\label{sec-111_2}
\section{2011-06-08}
\label{sec-112}
\subsection{GLZ 「ちょっとやな感じ連絡委員会」}
\label{sec-112_1}
\subsubsection{\textbf{DONE}}
\label{sec-112_1_1}

「ちょっとやな感じ連絡委員会」

TOKYO FM の DJは、自局の周波数を云うとき。

「80.0 (エイティ ポイント ラブ)」
                          \~{}~~~
とか云いやがるです。

いくら好きなDJの子でも、毎回軽くイラっとします。
\subsubsection{\textbf{DONE}}
\label{sec-112_1_2}

「ちょっとやな感じ連絡委員会」

電車内ではいろいろあります。

僕は背が高いほうではないので、
背が高い人が吊り革を持って背後に立つと、肘が頭に乗るときがあります。

他にも、背後に立つオヤジの「でっぱら」が背中にあたるのもちょっとイヤ。

あと、女性の後ろに立ったとき、普通にしているのにしきりに後ろを
気にする仕草をする人っていますよね。
「なんにもしてねーし、する気もねーよ」て心の中でつぶやきます。
\subsection{KIRAKIRA 私のちっちゃな心配事}
\label{sec-112_2}
\subsubsection{\textbf{DONE}}
\label{sec-112_2_1}

上京して早や20年。
こちらで家庭を持った自分ですが、未だ実家の私の部屋には自分の勉強机や本棚が
中身もそのまま残してくれてあります。

親たちは捨てるほどのものでもないという感覚かと思うのですが。
まあいつか掃除をと思っているのですけど、帰省はたいていドタバタしているので
なかなか実現できていない状況です。

ひとつ気になっているのが、本棚上の箱に入っているエロ本。
本棚自体移動したりしているので親は絶対箱の内容に気付いていると思うのですが
その件については何も云われておりません。

何度か持ってきてしまおうかなと思ったのですけど、
「あ、持ってったな」と思われるのもなんかイヤで。。。

実家近くに住んでいる姉夫婦の甥っ子が大きくなってきているので、彼に見つかる前には
回収したいと思ってはいるのですが。。。
\section{2011-06-07}
\label{sec-113}
\subsection{glz ナイスリカバー}
\label{sec-113_1}
\subsubsection{\textbf{DONE}}
\label{sec-113_1_1}

ちょっとアレな子どもだった僕は、片手にいっぱいくらいの大きさの石を当時まだあった木でできている電信柱に投げて遊んでいました。
おそらく当たったときの音を楽しんでいたのだと思いますが、何回が投げていた一発が柱の真芯に当たったらしく、まっすぐ僕の元へ戻ってきて頭に激突。
一瞬くらっとした後、タラリと顔面を血が滴り落ち、パニックになって家に帰りました。

それなりのコブが引いた後、生え際にハゲができまして。

中学に進学したとき、ウチのあたりはまだ坊主頭だったので、右眉の上あたりに小さな剃り込みが目立ちました。

で、30代になったくらいかなぁ、その剃り込みはキレイに消えました。
たぶんもっと大きなハゲになってても消えたよ。
\subsection{面白ヤンキー図鑑}
\label{sec-113_2}
\subsubsection{\textbf{DONE} ヽ(゚∀゚)ノ 5回目 アッサリだったが}
\label{sec-113_2_1}

面白ヤンキー図鑑

田舎にいたころ

『国際暴走族連合』

とガムテープに書いてスクーターの正面に貼ってたヤンキーがいた。
だいたい二人くらいでいつも走ってた。
\subsubsection{\textbf{DONE}}
\label{sec-113_2_2}

調子に乗ってバイクで派手なUターンしようとしたヤンキー。

ぐるりと一周してしまい。そのまま進行方向変えずに去って行きました。
\subsection{glz 屈辱ランキング}
\label{sec-113_3}
\subsection{KIRAKIRA おじいちゃん と おばあちゃん で2時間半}
\label{sec-113_4}
\section{2011-05-31}
\label{sec-114}
\subsection{KIRAKIRA こうやって覚えてます}
\label{sec-114_1}
\subsection{glz 心が死んでる人}
\label{sec-114_2}
\subsubsection{\textbf{DONE}}
\label{sec-114_2_1}

心が死んでる人

それはワタシ。
先月会社のファイルサーバ(書類など溜めておくところ) がクラッシュ。
データが2ヶ月ほど前の状態まで巻き戻ってしまった。

その場は特に支障もなかったのだけれど、今月になってから、自分の担当分の仕様書が
消えていることが判明。

作ったプログラムから追いかけて仕様書を作らなきゃいけない状態。
3ヶ月前の仕様なんて覚えてねーよ。
\subsubsection{\textbf{DONE}}
\label{sec-114_2_2}

トイレの個室から聞こえてくるイビキ
\subsection{同情ランキング}
\label{sec-114_3}
\subsubsection{\textbf{TODO}}
\label{sec-114_3_1}

同情ランキング

ピカピカの新品らしきバイクに颯爽とまたがった兄ちゃん。
踏切りで子どもをよけようとして転倒、ハンドルがポッキリと折れた。
子どもは気付かないまま踏み切りを越えて行ってしまった。

手に持ったハンドルを見つめて立ちつくしている。

という情景を目のあたりにしたとき心底兄ちゃんに同情した。
\section{2011-05-30}
\label{sec-115}
\subsection{glz キザニア}
\label{sec-115_1}
\subsubsection{\textbf{DONE}}
\label{sec-115_1_1}

「あたしのことどのくらい好き」と訊かれたら

「このくらい」と人差し指と親指を拡げて示す、

「えー、そんなにちょっとなの?!」と云うので

「地球はこのくらいだけどね」と指の間を3mmくらいにして見せる。
\section{2011-05-27}
\label{sec-116}
\subsection{K ホニャララ担当大臣なんです}
\label{sec-116_1}
\subsection{CIRCUSx2 五反田}
\label{sec-116_2}
\subsubsection{\textbf{DONE} ヽ(゚∀゚)ノ 1回目}
\label{sec-116_2_1}

渡部さん、こんにちは、いつも聴いております。

五反田は上京して初めて勤めた会社がありました。

通勤時間2時間弱。
電車自体にそんなに長い時間乗ったことが無かったので、この先やっていけるのかどうかたいへん不安になった覚えがあります。

五反田駅から見えた「ソープランド」の大きな看板を見て、「おお!東京だ」と思ったものです。

社屋の移転まで、結局 6、7年ほど通ったでしょうか。

さいきん、久しぶりに訪ずれたのですが、あまり変わっていない様子でした。
看板は無くなってましたが。。。

二十歳で上京した自分を育ててくれた街の一つです。
\subsubsection{\textbf{DONE}}
\label{sec-116_2_2}

さきほど読んでいただきました、ありがとうございます。

ふいに思い出したのですが、印象深い食べ物屋さんに「ステーキハウス リベラ」がありました。
店内はプロレスラーの写真がところ狭しと貼ってあります。レスラーや格闘家たちの御用達なのだとか。

何度かお邪魔したのですが、そういった方たちとは遭遇しませんでした、残念。
久しぶりにがっつり肉でも食べたくなりました。

普段はまったく、肉より魚なんですが。

魚と云えば同じ通りに「オイスターバー」もありますね。

って、食べ物屋さんのことばっかり思い出しますね。
週末、ヨメサンとチビたち連れて行ってみようかな。
\subsection{}
\section{2011-05-26}
\label{sec-117}
\subsection{KIRAKIRA かためた かたまった}
\label{sec-117_1}
\subsection{役得大全集}
\label{sec-117_2}
\section{2011-05-25}
\label{sec-118}
\subsection{若さ炸裂!}
\label{sec-118_1}
\subsection{真の男の節電対策}
\label{sec-118_2}
\subsection{男の575}
\label{sec-118_3}
\subsubsection{\textbf{DONE}}
\label{sec-118_3_1}

ふんどし 575
汗しぶき おれとお前と ふんどしと
\subsubsection{\textbf{TODO}}
\label{sec-118_3_2}

男の575

おさまらねぇ 
\section{2011-05-24}
\label{sec-119}
\subsection{煩悩ラジオ}
\label{sec-119_1}
\subsubsection{\textbf{DONE}}
\label{sec-119_1_1}

煩悩ラジオ

保健室の先生、石原さとみにお腹サスサスされてぇ!
「おなか痛いの?」とか云われてぇ!!!
\subsubsection{\textbf{DONE}}
\label{sec-119_1_2}

歯科助手の石原さとみに、ほじほじされてぇ!!!
\subsubsection{\textbf{TODO}}
\label{sec-119_1_3}

満員電車
\section{2011-05-23}
\label{sec-120}
\subsection{すごい相談コンテスト}
\label{sec-120_1}
\subsection{歳を取ったと感じる瞬間ランキング}
\label{sec-120_2}
\subsubsection{\textbf{DONE}}
\label{sec-120_2_1}

体の左右の動きが違うと思うようになった
\subsubsection{\textbf{DONE}}
\label{sec-120_2_2}

固いものを噛むとき、ちょっと度胸がいる
\subsubsection{\textbf{DONE} ヽ(゚∀゚)ノ 4回目}
\label{sec-120_2_3}

新しい本を読んでいて、ページがめくれないときがある。

指に唾をつけるのに抵抗が無くなる。
\section{2011-05-1}
\label{sec-121}
\subsection{GLZ イタさ炸裂}
\label{sec-121_1}
\subsubsection{\textbf{DONE}}
\label{sec-121_1_1}

イタさ炸裂

「イタさ炸裂」と云えばやはり子どもの名前でしょうと、ヨメサンに訊いてみると。
そりゃたくさんいるけど、やはりこの三兄弟でしょう、とのこと。

  長男:琉玖(ルーク)
  次男:琉音(りおん)
  三男:来夢(らいむ)

勿論、日本人です。僕も会ったことありますが、ふつーの子たちです。
子どもたちに罪はありません。親にも無ければ良いのだけれど。

あと、さいきん知った子で
「凛優(りゅう」くん、ううむ。
学校の先生は大変だ。

P.S.
もっと教えてよと云ったら「友だちを売るみたいでいやだ」と拒否されました。
キミの友だちを僕に売りなよ、と云ったら殴られました。
\subsubsection{\textbf{TODO}}
\label{sec-121_1_2}

イタさ炸裂

飲んでる大学生たち

居酒屋となりの樽。
恋愛論
「あたしなら付き合わない
\section{2011-05-18}
\label{sec-122}
\subsection{ダメマシーン}
\label{sec-122_1}
\subsubsection{\textbf{DONE}}
\label{sec-122_1_1}

10年以上前、ROVER MINIが好きで乗っていたのですが。
新車でしたが外国からの輸入のものだったからなのか、しばしばトラブっていました。

長期休暇のおり、帰省のためMINIに乗り、真夜中の長いドライブに出掛けました。

さて、高速の入口が近付き車線変更をしようとウィンカーを出したのですが、「カチカチ」と音がするだけでウィンカーが点滅しているようには見えません。
あわてて何度かスイッチを操作したのですがやはり点きません。

仕方無く、帰省はあきらめて、そのまま家にUターンしました。
高速入口から家まではウィンカーの代わりに窓を開けて(もちろん手動で開け) 手信号をしながら道を曲がって帰りました。むなしい\ldots{}

高速道路に乗る前で良かった、中に入ってしまってからでは窓を開けて手信号出すわけにもいかず面倒なことになったかと思われます。
\subsection{男と女のこわいランキング2}
\label{sec-122_2}
\subsubsection{\textbf{DONE}}
\label{sec-122_2_1}

どうやら彼女が、ボクの髭剃りで、
ワキやすねを処理しているらしい形跡を見つけた。
\subsubsection{\textbf{DONE} ヽ(゚∀゚)ノさいよー 3回目 でも0だった}
\label{sec-122_2_2}

男と女のこわいランキング2
 
すでに浮気していることがばれていて、夫婦間はひえひえの状態。

ある日の夜、遅く帰ると鍵が開いていない。
浮気の引け目もあり恐る恐る鍵を開けて家に入ると中は真っ暗。すでに妻は寝ているようだ。

自分の部屋へ爪先立ちで廊下を移動。
部屋へ入るとなんとなく安堵の溜め息をついて電灯のスイッチをオンにした。

 自分の机の上にフォークが突き刺さっていた。

木製の机とはいえ、けっこう固い材質の表面はためらいキズもなく、一発必中。

「どう考えても予行演習だよな」と背筋が凍ったそうです。

友人の本当の話です。
これを聞いた当時は独身だったのですが、(結婚なんてするもんじゃねぇな) と本気で思いました。
包丁でもナイフでもなく、フォークだったというのが大変恐しかった。。。

P.S. 彼はその後無事に浮気相手と再婚しましたとさ。
\subsubsection{\textbf{DONE}}
\label{sec-122_2_3}

男と女のこわいランキング2

一方的に振った彼女と、友だちたちの飲み会で再開。

なるべく近付かないようにしていたが。
酔っぱらってしまい、いい気持ちでいたら。

いつの間にか隣に座っていた彼女、

「あとでちょっと話があるんだけど」

とすこし丸みをおびたおなかをさすりながら云われたとき。。。。
\section{2011-05-17}
\label{sec-123}
\subsection{KIRAKIRA やぶった、やぶられた}
\label{sec-123_1}
\subsubsection{\textbf{DONE}}
\label{sec-123_1_1}

子どもの頃、私が破ったのは床。

古い家だったので上でドタバタしすぎて床を抜きました。
数週間前に床下浸水した影響もあったと思います。

干したあとの布団があったので、その上で前転宙返りの練習を始めた私。
調子よくなってきて、何度も何度も床に「ドスン!」と落ちました。

夜、コタツに入るとなぜか冷気が下から来ます。中を見ればぽっかりと穴が空いている状態。
昔の掘りゴタツでしたので炭、豆炭などを入れる「堀り」の部分が衝撃により壊れ、ごそっと下に落ちてしまっていました。
そして母が気付いたのですが、どうも床を踏むといつもと違う感覚があり。私が暴れていたあたりの床がふわふわしてて、踏むと床が沈み込む状態。床下の支えている柱がパッキリと折れてました。

次の日、大工の父がササっと直してくれたのですが、こっぴどく怒られたのは云うまでもありません。
\subsection{GLZ 私のいい所発表}
\label{sec-123_2}
\subsection{男と女の怖いランキング}
\label{sec-123_3}
\subsubsection{\textbf{DONE} ヽ(゚∀゚)ノ 2回目よまれた}
\label{sec-123_3_1}

ぴあすネタ。読まれた
\section{2011-05-12}
\label{sec-124}
\subsection{KIRAKIRA ぬりぬりパタパタ メイク・お化粧の話}
\label{sec-124_1}
\subsubsection{\textbf{DONE}}
\label{sec-124_1_1}

小島さん、瀧さん、こんにちは。二度寝は気をつけましょう

「ぬりぬりパタパタ メイク・お化粧の話」

ヨメサンが週末友だちの結婚式とのことで、ご近所の式場でメイクをされているお友だちに当日メイクしてもらうとのこと。

うちのヨメサンはいつもほぼスッピンで過ごしていて、出かけるときにちょいちょいとイジる感じで。あまりガッツリ化粧をしているのを見たことがありません。
ガッツリメイク、で思い出すのは我々の結婚式のときのこと、式場のメイクの方にお願いしたら、あらまぁビックリ、かわいいじゃないの。

式場で担当してくださった方にも「いつもはあんまりお化粧されないんですね」と云われてたので女性から見てもずいぶんな変わりようだったみたいです。

5歳のチビも結婚式のときのヨメサンの写真を見ていろいろ思うところがあるようです。

それで今週末。メイクの仕上がりが今から楽しみです。

「もっとメイクの勉強をしなよ」と云いたいときもありますが、まぁスッピンの方がヨメサンらしいので何年かに一度の楽しみにしておきます。
\subsection{glz おじいちゃんおばあちゃんレベルの感心する事}
\label{sec-124_2}
\subsection{ひとこと劇場 おしい英語編}
\label{sec-124_3}
\subsubsection{\textbf{TODO}}
\label{sec-124_3_1}

ペットボトル>ペットホテル
\subsubsection{\textbf{DONE}}
\label{sec-124_3_2}

ひとこと劇場 おしい英語編

「病院であれ入れられたときは本当に痛かった、ええとアレ、『カレーテル』?」
\subsubsection{}
\section{2011-05-10}
\label{sec-125}
\subsection{オッサンイミダス2011}
\label{sec-125_1}
\subsubsection{\textbf{TODO}}
\label{sec-125_1_1}

フロッピー

芸能人
\subsubsection{\textbf{DONE}}
\label{sec-125_1_2}

オッサンイミダス2011

小学生の頃、突然もみあげを剃りあげてきた子がいた。
刈り上げではない普通の長めの髪なのに、サイドだけ青々してきて、すっごいヘン。
どうやらテクノカットのつもりだったらしい。YMOの音楽が拡がり始めた頃のこと。
もみあげだけ落としても面白い頭になるだけなんだけど、まぁ子どものやることですから。。。。

もみあげをスパッと切り落としたヘアスタイルが流行ったのはその一時期だけでしたでしょうか。
流行りすぎて、禁止にされた学校もあったとか。

> ちがうな。。。|ω・)
\subsubsection{\textbf{DONE}}
\label{sec-125_1_3}

オッサンイミダス2011

アベックと云うとヨメサンにおこられます。
\subsubsection{\textbf{TODO}}
\label{sec-125_1_4}


ズロース、シミーズは前の世代です。
\subsubsection{\textbf{DONE}}
\label{sec-125_1_5}

オッサンイミダス2011

キックボードが流行ったとき、「ローラースルーゴーゴー」を思い出しました。
\subsubsection{\textbf{DONE}}
\label{sec-125_1_6}

オッサンイミダス2011

チャンネル「回して」と云う。回すもんですあれは。
\subsubsection{\textbf{DONE}}
\label{sec-125_1_7}

オッサンイミダス2011

「ズックはきなさい」と子どもにすりこんでます
\subsubsection{\textbf{DONE}}
\label{sec-125_1_8}

オッサンイミダス2011

コーヒー牛乳作ってと云うと「カフェオレ!」。
一緒だろ。
\subsubsection{\textbf{DONE}}
\label{sec-125_1_9}

オッサンイミダス2011

ジーパン。
\subsubsection{\textbf{DONE}}
\label{sec-125_1_10}

オッサンイミダス2011

「ちょっと擦りむいちゃったから、赤チン取って」
\subsubsection{\textbf{DONE}}
\label{sec-125_1_11}

オッサンイミダス2011

「ほらほら、早くチャック上げて、出かけるよ」などとチビに云ってました。
もう「チャック」て云わないんすかね。
\subsubsection{\textbf{DONE}}
\label{sec-125_1_12}

オッサンイミダス2011

「あれあれあれー! 社会の窓が開いてるよー」
\subsection{値崩れ祭}
\label{sec-125_2}
\subsubsection{\textbf{TODO}}
\label{sec-125_2_1}

東電。。。
\subsubsection{\textbf{TODO}}
\label{sec-125_2_2}

値崩れ祭

思い出すのはやっぱり自分のMacを初めて買ったとき。
最上位機種で、ディスプレイ、キーボードなど一揃い購入したら100万円を越えた。

しかし次の最上位機種はとつぜん30万まで安くなった、相当売れなかったのか。
\section{2011-05-09}
\label{sec-126}
\subsection{ゴールデンウィークもう二度と行かない所}
\label{sec-126_1}
\subsubsection{\textbf{DONE}}
\label{sec-126_1_1}

いまもっとも行きたくない所… 高速道路。

僕、ヨメサンとチビたちで。5月3日の朝に実家へ向うための高速道路に乗りました。

チビたちはオムツ履いていたので、基本まったく平気だったのですが。大人たちはちょこちょこパー
キングに入らねばならず。プラス、チビたちのニーズにも応えるための寄り道も多く。
けっきょく夕食も高速道路のS.A.で摂り、到着は夜となりました。

覚悟はしていたものの普段であれば4時間で到着する道程が 9時間 かかりました。

ちなみに子ども用のオムツがあれば大人でも用を足すことができるという裏ワザがある
ことを、実家方面に集結したイトコから聴きました。
\subsubsection{\textbf{DONE}}
\label{sec-126_1_2}

昨日GW最後の日ということで家族で行き慣れた駅で降り、買い物などをしていました。
母の日ということもあり、ランチはいつものデパート上階の通い慣れたフロアへ行くのはやめて、少し歩いたところにあるイタリアンレストランへ来ました。

しかし、その、こじゃれたレストランは結婚式の披露宴だかを行なっているらしく貸切。
仕方なく、すぐ隣にある同系列っぽいカフェに入ったのですが。高いわ、出てくるのが遅いわ、とくに旨くもなく。量も少ない。
われわれあんまりそういうことに厳しくない夫婦なのですが、「これヒドイね」「あんまりだね」と小声でぼそぼそぼそ。
5歳のチビは「ハンバーグ、ウマイー」と云っていたのですが、すぐ量が足りなくなり「もっとー」。こんな高いマズイところで追加できるかい! とすぐ会計を済ませて、デパートまで戻っていつものお店でケーキを頼みました。

二度と行・か・な・い!
\subsection{海外575}
\label{sec-126_2}
\subsubsection{\textbf{TODO}}
\label{sec-126_2_1}

そのハラで ビキニを着ても いいんだね

たびホテル 灰皿ひとつ もらえない
ひたすらに 作り笑顔で エーゴ聞く
太った子 ハーゲンダッツ ベッドで喰う
\section{11.05.02(mon)-12:51}
\label{sec-127}
\subsection{ゴールデンウィーク中間報告}
\label{sec-127_1}
\subsubsection{\textbf{:glz:}}
\label{sec-127_1_1}

今日仕事してます、明日から家族で僕の実家に帰省。
\subsubsection{\textbf{DONE}}
\label{sec-127_1_2}

「ゴールデンウィーク中間報告」 (すみ

昨日やっと、わが家でもっともコストパフォーマンスが良くない、ガラスケース入りの兜と人形を飾りました。大きくはないのですが鯉のぼりもベランダへ。

日の目を見る期間が数日のクセに押入の場所を大きく取られるのが、僕もヨメサンも納得行っていません。

が、両親たちからの贈りものなのでもちろん処分するわけにも行かず。
子どもたちが「飾ってー」と云ってくるうちは置いときますがー。
\subsubsection{\textbf{DONE}}
\label{sec-127_1_3}

今日は普通に出勤です。6日はなんとか有休をもらいました。
周りは働きバチばかりでGWに休もうという雰囲気があまりなく困っています。
子持ちは普通に休みたいんだよ連休とかっ!
\subsubsection{\textbf{TODO}}
\label{sec-127_1_4}

明日から帰省しますのでラジオ聴けません。
\subsubsection{\textbf{DONE}}
\label{sec-127_1_5}

ピストンさん

第48回ギャラクシー賞
DJパーソナリティ賞 受賞おめでとうございます。

授賞式でなにやらかしましょうか?
\subsubsection{\textbf{DONE}}
\label{sec-127_1_6}

BBQ、チビたち連れて行きたかったのですが、5日のその時間はおそらく渋滞に揉まれて
います\ldots{} ザンネン。
またやってください。
\section{11.04.28(thu)-10:47}
\label{sec-128}
\subsection{ありがち外国人}
\label{sec-128_1}
\subsubsection{}

欧米の人、太ったオバさんなのに、胸元ガバッ て開いたタンクトップ着てる。 (済み
***
乳トップを気にしない (ヨシッ  (済み
\subsubsection{}

スーツがなぜか窮屈そう。日本で買うとそうなるのか? 骨格? (すみ
\subsubsection{}

彫りが深いというだけでカッコ良く見える、ズルイ。(すみ
\subsubsection{}

中肉中背の人がいない
\subsubsection{}

着物の人を見ると「ふぁんたすてぃっく!」 (済み
\subsubsection{}

深い刈り上げは中国人
\subsubsection{}

HAHAHAHAHAHA! てホントに笑う。(すみ
\subsubsection{}

今、海外行ったら、日本人! 放射能がうつる! とか云われそう
\subsubsection{}

時代劇に出てくる外国人は、ほんとにいつもあんな格好をしていたのだろうか。普段着
もあるだろうに。(すみ
\subsubsection{}
\section{11.04.27(wed)-16:45}
\label{sec-129}
\subsection{インチキ料理大集合}
\label{sec-129_1}
\subsubsection{}

某有名チェーンラーメン店にて「太肉(ターロー)麺」的なものを頼んだとき。
スタッフが真空パックのターローを開けていた。
納得行かなかった。 【済み
\subsubsection{}

そもそも普通に料理をするので、インチキというのが思いつかない、、、
ので、考えてみた。
\subsubsection{}

お椀に鰹だしと味噌、具をなんでも入れてお湯を注げばみそ汁。
みりんをちょろりと入れればインチキとは思われないです。
\subsubsection{}

市販のハンバーグなど焼いた後の肉汁が残っているところへ
ウスターソース、ケチャップ、酒かワイン、バター、塩、こしょう、砂糖をがっと入れ
てぐつぐつしたものをハンバーグにかけるインチキデミグラスソース。 済み
\section{11.04.21(thu)-10:22}
\label{sec-130}
\subsection{黒い爽快感}
\label{sec-130_1}
\subsubsection{電車、バス、でんわ とか。えろは三浦フィルタあんまり通んない}
\label{sec-130_1_1}


まいったなぁ、パンツ見えちゃったよ

 まだ口パクパクしている魚に包丁を入れるとき、
(ごめんなぁ) とちょっと思いつつ一気に刃先を入れる

 電車内、カツラだなぁて人の前で足元に注意を向けさせる。視線を下げさせ
 て頭が揺れたときその真偽を審議する。正解すると爽快です。

 薄毛を隠した感じのヘアスタイルの人に風を送ってみる

 薄毛を隠した感じのヘアスタイルの人を背に 「あつい〜」とウチワで

 
高校時代の基本おちゃらけキャラの同級生は、高3の就職進学を決める時期、まったくおふざけで自衛隊に入隊希望の書類を出したら、あれよあれよという間に入隊が決定してしまった。
彼は真っ青になったが、そのまま自衛官の道へ。
数年後めちゃくちゃマッチョになった彼に会うことになった。
人生怖いと思った、いいネタもらったなとも思った。
\subsection{kirakira 豆のはなし}
\label{sec-130_2}
\section{Help}
\label{sec-131}

C-c C-t TODO->DONE->none

\end{document}
